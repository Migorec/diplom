\documentclass[utf8x, 14pt]{G7-32} % Стиль (по умолчанию будет 14pt)

% Остальные стандартные настройки убраны в preamble-std.tex
\include{preamble-std}
\usepackage{pdfpages}
\usepackage{longtable, multirow, rotating, color, colortbl}
\usepackage{amsmath}
\usepackage{tikz}
\usepackage{pgfplots}
\begin{document}
\thispagestyle{empty} 
\begin{center}
Рецензия

на квалификационную работу Миникса И.В.

<<Разработка системы имитационного моделирования в форме библиотеки языка Haskell>>

\end{center}

Работа Миникса И.В. посвящена проблеме имитационного моделирования систем массового обслуживания. ЕЩЕ ПРЕДЛОЖЕНИЕ. Структурно работа состоит из введения, аналитического раздела, конструкторскоого раздела, технологчского раздела, организационно-экономического раздела, раздела по охране труда и заключения. 

В работе описывается использования механизма монад для создания в рамках языка Haskell языка описания систем, схожего с языком, использующимся в системе GPSS. Был разработан и реализован алгоритм имитационного моделирования описанных систем. В рамках тестирования разработанной библиотеки построена аналитическая модель тестовой системы массвого обслуживания и эксперементально подтверждено соответствие результатов полученных путем имитационного моделирования с теоретически ожидаемым.

Достоинствами работы являются кросплатформенность разработанного по и 

Из недостатков рассматриваемой работы следует отметить отсутствие в реализованном ПО встроенных средств статистического анализа.

Несмотря на указанные недостатки, можно заключить, что данная работа отвечает требованиям, предъявляемым к выпускной квалификационной работе, а ее автор Миникс И.В. заслуживает отличной оценки и присвоения квалификации дипломированноого специалиста.

\vspace{1cm}

Звание должность \hspace{5cm} Фамилия И.О

\end{document}
