\chapter{Организационно-экономический раздел}

Организационно-экономическая часть процесса разработки программного продукта предусматривает выполнение следующих работ:
\begin{itemize}
\item формирование состава выполняемых работ и группировка их по стадиям разработки;
\item расчет трудоемкости выполнения работ;
\item установление профессионального состава и расчет количества исполнителей;
\item определение продолжительности выполнения отдельных этапов разработки;
\item построение календарного графика выполнения разработки;
\item контроль выполнения календарного графика.
\end{itemize}

\section{Формирование состава выполняемых работ и группировка их по стадиям разработки}

Разработку программного продукта можно разделить на следующие стадии:

\begin{itemize}
\item техническое задание;
\item расчет трудоемкости выполнения работ;
\item эскизный проект;
\item технический проект;
\item рабочий проект;
\item внедрение.
\end{itemize}

Допускается объединение технического и рабочего проекта в технорабочий проект.

Планирование длительности этапов и содержания проекта осуществляется в соответствии с ЕСПД ГОСТ 34.603--92 и распределяет работы по этапам. как показано в таблице \ref{tab:jobsAndStages}.

\newpage
\footnotesize
\begin{longtable}{|l|c|p{0.65\textwidth}|}
    \caption{Распределение работ проекта по этапам}
    \label{tab:jobsAndStages}
        \\ \hline
        \multicolumn{1}{|l|}{\centering Основные стадии}
      & \multicolumn{1}{|c|}{\centering \No}
      & \multicolumn{1}{|p{0.5\textwidth}|}{\centering Содержание  работы} \\
        \hline
            \endfirsthead
        
        \subcaption{\normalsize Продолжение таблицы~\ref{tab:jobsAndStages}}
        \\ \hline \endhead
        \subcaption{\normalsize Продолжение на следующей странице}
        \endfoot
        \hline
        \endlastfoot
        
        \multirow{2}{*}{\centering 1. Техническое задание} & 1 & Постановка задачи \\
        \cline{2-3}
        & 2 & Выбор средств проектирования и разработки \\
        \hline
        \multirow{4}{*}{\centering 2. Эскизный проект} & 3 & Разработка структуры системы \\
        \cline{2-3}
        & 4 & Разработка алгоритмов описания моделей и моделирования \\
        \cline{2-3}
        & 5 & Разработка вспомогательных алгоритмов \\
        \cline{2-3}
        & 6 & Разработка пользовательского интерфейся \\
        \hline
        \multirow{7}{*}{\centering 2. Технорабочий проект} & \centering 7 & Реализация алгоритмов описания моделей и моделирования\\
        \cline{2-3}
        & 8 & Реализация вспомогательных алгоритмов \\
        \cline{2-3}
        & 9 & Реализация пользовательского интерфейса \\
        \cline{2-3}
        & 10 & Отладка программного продукта \\
        \cline{2-3}
        & 11 & Исправление ошибок и недочетов \\
        \cline{2-3}
        & 12 & Разработка документации к системе \\
        \cline{2-3}
        & 13 & Итоговое тестирование системы \\
        \hline
        4. Внедрение & 14 & Установка и настройка программного продукта \\
        \hline
\end{longtable}
\normalsize

\section{Расчет трудоемкость выполнения работ}

Трудоемкость разработки программной продукции заывисит от ряда факторов, основными из которых являются следующие:

\begin{itemize}
\item степень новизны разрабатываемого программного продукта;
\item сложность алгоритма его функционирования;
\item объем используемой информации, вид ее представления и способ обработки;
\item уровень используемого алгоритмического языка программирования.
\end{itemize}

Разрабатываемый программный продукт можно отнести:

\begin{itemize}
\item по степени новизны~--- к категории В. Разрботка программной продукции имеющей аналоги. 
\item по степени сложности алгоритма функционирования~--- к 1-ой группе (программная продукиця реализующая моделирующие алгоритмы). 
\end{itemize}

Трудоемкость разработки программного продукта $\tau_{\text{ПП}}$ может быть определена как сумма величин трудоемкости выполнения отдельных стадий разработки ПП из выражения~\ref{F:tayPP}.

\begin{equation}
\tau_{\text{ПП}} = \tau_{\text{ТЗ}} + \tau_{\text{ЭП}} + \tau_{\text{ТП}} + \tau_{\text{РП}} + \tau_{\text{В}}
\label{F:tayPP}
\end{equation}, где

$\tau_{\text{ТЗ}}$~--- трудоемкость разработки технического задания; $\tau_{\text{ЭП}}$~--- трудоемкость разработки эскизного проекта; $\tau_{\text{ТП}}$~--- трудоемкость разработки технического проекта; $\tau_{\text{РП}}$~--- трудоемкость разработки рабочего проекта; $\tau_{\text{В}}$~--- трудоемкость внедрения.

Трудоемкость разработки технического задания рассчитывается по формуле~\ref{F:tayTZ}

\begin{equation}
\tau_{\text{ТЗ}} = T_{\text{ЗРЗ}} + T_{\text{ЗРП}}
\label{F:tayTZ}
\end{equation}, где

$T_{\text{ЗРЗ}}$~--- затраты времени разработчика постановки задач на разработку ТЗ, чел.-дни; $T_{\text{ЗРП}}$~--- затраты времени разработчика ПО на раззработку ТЗ, чел.-дни.

Значения величин $T_{\text{ЗРЗ}}$ и $T_{\text{ЗРЗ}}$ рассчитываются по формулам \ref{F:TZRZ} и \ref{F:TZRP}.

\begin{equation}
T_{\text{ЗРЗ}} = t_{\text{З}} \cdot K_{\text{ЗРЗ}}
\label{F:TZRZ}
\end{equation}

\begin{equation}
T_{\text{ЗРП}} = t_{\text{З}} \cdot K_{\text{ЗРП}}
\label{F:TZRP}
\end{equation}, где

$t_{\text{З}}$~--- норма времени на разработку ТЗ на ПП в зависимости от его функционального назначенияя и стпени новизны, чел.-дни; $K_{\text{ЗРЗ}}$~--- коэффициент, учитывающий удельный вес трудоемкости работ, выполняемых разработчикм ТЗ; $K_{\text{ЗРЗ}}$~--- коэффициент, учитывающий удельный вес трудоемкости работ, выполняемых разраьотчиком ПО на стадии ТЗ.

$t_{\text{З}} = 24$~чел.-дн. (управление НИР)

$K_{\text{ЗРЗ}} = 0.65$ (совместная разработка)

$K_{\text{ЗРП}} = 0.35$ (совместная разработка)

$\tau_{\text{ТЗ}} = 24 \cdot 0.65 + 24 \cdot 0.35 = 24$~чел.-дн.

Аналогично рассчитывается трудоемкость эскизного проекта $\tau_{\text{ЭП}}$:

$\tau_{\text{ЭП}} = 70 \cdot 0.5 + 70 \cdot 0.5 = 70$~чел.-дн.

Трудоемкость разработки технического проекта $\tau_{\text{ТП}}$ зависит от функционального назначения ПП, количества разновидностей форм входной и выходной информации и определяется как сумма времени, затраченного разрабьотчикм постановки задач и разработчиком программного обеспечения по формуле~\ref{F:tauTP}.

\begin{equation}
\tau_{\text{ТП}} = (t_{\text{ТРЗ}} + t_{\text{ТРП}}) \cdot K_{\text{В}} \cdot K_{\text{Р}}
\label{F:tauTP}
\end{equation}, где

$t_{\text{ТРЗ}}$ и $t_{\text{ТРП}}$~--- норма времени, затрачивваемого на разработку ТП разрабьотчиком постановки задач и разработчиком программного обуспечения соответственно, чел.-дни; $K_{\text{В}}$~--- коэффициент учета вида используемой информации, $K_{\text{Р}}$~--- коэффициент учета режима обработки информации.

Значение коэффициента $K_{\text{В}}$ определяется из выражения:

\begin{equation}
K_{\text{В}} = \frac{K_{\text{П}} \cdot n_{\text{П}} + K_{\text{НС}} \cdot n_{\text{НС} + K_{\text{Б}} \cdot n_{\text{Б}}}}{n_{\text{П}} + n_{\text{НС}} + n_{\text{Б}}}
\end{equation}

где  $K_{\text{П}}$, $K_{\text{НС}}$, $K_{\text{Б}}$~--- значения коэффициентов учета вида используемой информации для переменной, нормативно-справочной информации и баз данных соответственно; $n_{\text{П}}$, $n_{\text{НС}}$, $n_{\text{Б}}$~--- количество наборов данных переменной, нормативно-справочной информации и баз данных соответственно.

$K_{\text{П}} = 1$, $K_{\text{НС}}$ - 0.72, $K_{\text{Б} = 2.08}$

$K_{\text{В}} = \frac{1 \cdot 3 + 0.72 \cdot 1 + 2.08 \cdot 0}{3 + 1 + 0} = 0.505$

$K_{\text{Р}} = 1.26$

$\tau_{\text{ТП}} = (33 + 10) \cdot 0.505 \cdot 1.26 = 28$, чел.-дн.

Трудоемкость разработки рабочего проекта $\tau_{\text{РП}}$ зависит от функционального назначения ПП, количества разновидностей форм входной информации, сложности алгоритма функционирования, сложности контроля информации, степени использования готовых программных модулей, уровня алгоритмического языка программирования и определяется по формуле:

\begin{equation}
\tau_{\text{РП}} = K_{\text{К}} \cdot K_{\text{Р}} \cdot K_{\text{Я}} \cdot K_{\text{З}} \cdot K_{\text{ИА}} \cdot (t_{\text{РРЗ}} + t_{\text{РРП}})
\label{F:tauRP}
\end{equation}

где $K_{\text{К}}$~--- коэффициент учета сложности контроля информации; $K_{\text{Я}}$~--- коэффициент учета уровня используемого алгоритмического языка программирования; $K_{\text{З}}$~--- коэффициент учета степени использования готовых программных модулей; $K_{\text{ИА}}$~--- коэффициент учета вида используемой информации и сложности алгоритма ПП.

$K_{\text{К}} = 1, K_{\text{Р}} = 1.44, K_{\text{Я}} = 1, K_{\text{З}} = 0.7, t_{\text{РРЗ}} = 9$~чел.-дн., $t_{\text{РРП}} = 54$~чел.-дн., $K_{\text{П}} = 1.2, K_{\text{НС}} = 0.65, K_{\text{Б}} = 0.54$

$K_{\text{ИА}} = \frac{1.2 \cdot 3 + 0.65 \cdot 1 + 0.54 \cdot 0}{3 + 1 + 0} = 1.06$

$\tau_{\text{РП}} = 1 \cdot 1.44 \cdot 1 \cdot 0.7 \cdot 1.06 \cdot (9 + 54) = 67$

Так как при разработке ПП стадии <<Технический проект>> и <<Рабочий проект>> объеденины в стадию <<Техно-рабочий проект>>, то трудоемоксть ее выполнения $\tau_{\text{ТРП}}$ определяется по формуле:

\begin{equation}
\tau_{\text{ТРП}} = 0.85 \cdot (\tau_{\text{ТП}} + \tau_{\text{РП}})
\label{F:tauTRP}
\end{equation}
$\tau_{\text{ТРП}} = 0.85 \cdot (28 + 67) = 91$


Трудоемкость выполнения стадии внедрения $\tau_{\text{В}}$ может быть расчитана по формуле:

\begin{equation}
\tau_{\text{В} = (t_{\text{ВРЗ}} + t_{\text{ВРП}}) \cdot K_{\text{К}} \cdot K_{\text{Р}} \cdot K_{\text{З}}}
\label{F:tauV}
\end{equation}

где $t_{\text{ВРЗ}}$, $t_{\text{ВРП}}$~--- норма времени, затрачиваемого разработчиком постановки задач и разработчиком ПО соответственно на выполнение процедур внедрения ПП, чел.-дни.

$\tau_{\text{В}} = (10 + 11) \cdot 1 \cdot 1.26 \cdot 0.7 = 19$ чел.-дн.

Подставив полученные данные в формулу~\ref{F:tayPP} получим:

$\tau_{\text{ПП}} = 24 + 70 + 91 + 19 = 204$ чел.-дн.

\begin{table}[ht]\footnotesize
    \caption{Распределение трудоемкости по стадиям разработки проекта}
    \begin{tabular}{|c|c|c|p{0.70\textwidth}|c|}
    \hline
    \begin{sideways}Этап\end{sideways} &
    \begin{sideways} \parbox{30mm}{Трудоемкость \\этапа, чел.-дн. }\end{sideways} &
    \No & \multicolumn{1}{p{0.7\textwidth}|}{\centering Содержание работы}&
    \begin{sideways} \parbox{30mm}{Трудоемкость, чел.-дн. }\end{sideways} \\
    \hline
    \multirow{2}{*}{\centering 1} & \multirow{2}{*}{\centering 24} & 1 & Постановка задачи, разработка ТЗ & 20 \\
    \cline{3-5}
    & & 2 & Выбор средств проектирования и разработки & 4\\
    \hline
    \multirow{4}{*}{\centering 2} & \multirow{4}{*}{\centering 70} & 3 & Разработка структуры системы & 15 \\
    \cline{3-5}
    & & 4 & Разработка алгоритмов описания модели и моделирования & 30\\
    \cline{3-5}
    & & 5 & Разработка вспомогательных алгоритмов & 15\\
    \cline{3-5}
    & & 6 & Разработка пользовательского интерфейса & 10\\
    \hline
    \multirow{7}{*}{\centering 3} & \multirow{7}{*}{\centering 91} & 7 & Реализация алгоритмов описания модели и моделирования & 23 \\
    \cline{3-5}
    & & 8 & Реализация вспомогательных алгоритмов & 16\\
    \cline{3-5}
    & & 9 & Реализация пользовательского интерфейса & 10\\
    \cline{3-5}
    & & 10 & Отладка программного продукта & 12\\
    \cline{3-5}
    & & 11 & Исправление ошибок и недочетов & 15\\
    \cline{3-5}
    & & 12 & Разработка документации к системе & 8\\
    \cline{3-5}
    & & 13 & Итоговое тестирование системы & 7\\
    \hline
    4 & 19 & 14 & установка и настройка ПП & 19 \\
    \hline
    & & &{\raggedleft{Итого:}} & 204 \\
    \hline
    \end{tabular}
\end{table}

\normalsize

\section{Расчет количества исполнителей}

Средняя численность исполнителей при реализации проекта разработки и внедрения ПО определяется соотношением:

\begin{equation}
N = \frac{Q_{\text{Р}}}{F}
\label{F:N}
\end{equation}

где $Q_{\text{Р}}$~--- затраты труда на выполнение проекта (разработка и внедрение ПО); $F$~--- фонд рабочего времени.

Величина фонда рабочего времени определяется соотношением:

\begin{equation}
F = T \cdot F_M
\label{F:F}
\end{equation}

где $T$~--- фвремя выполнения проекта в месяцах, равное 4 месяцам; $F_M$~--- фонд времени в текущем месяце, который рассчитывается из учета числа дней в году, числа выходных и праздничных дней:

\begin{equation}
F_M = \frac{t_{\text{р}} \cdot (D_{\text{К}} - D_{\text{В}} - D_{\text{П}})}{12}
\label{F:FM}
\end{equation}

где $t_{\text{р}}$~--- продолжительность рабочего дня; $D_{\text{К}}$~--- общее число дней в году; $D_{\text{В}}$~--- число выходных дней в году; $D_{\text{П}}$~--- число праздничных дней в году.

$F_M = \frac{8 \cdot (365 - 103 - 10)}{12} = 168$

$F = 4 \cdot 168 = 672$

$N = \frac{204 \cdot 8}{672} = 3$~--- число исполнителей проекта.

\section{Календарный план-график}

\begin{table}[ht]\footnotesize
\caption{Планирование разработки}
\begin{tabular}{|p{0.2\textwidth}|l|p{0.25\textwidth}|p{0.17\textwidth}|l|}
\hline
Стадия разработки & Трудоемкость & Должность исполнителя& Распределение трудоемкости & Численность \\
\hline
\multirow{2}{0.2\textwidth}{Техническое задание} & \multirow{2}{*}{24} & Ведущий программист & 18(75\%) & 1\\
& & Программист 1 & 6(25\%) & 1 \\
\hline
\multirow{3}{0.2\textwidth}{Эскизный проект} & \multirow{3}{*}{70} & Ведущий программист & 26(37\%) & 1\\
& & Программист 1 & 27(39\%) & 1 \\
& & Программист 2 & 17(24\%) & 1 \\
\hline
\multirow{3}{0.2\textwidth}{Технорабочий проект} & \multirow{3}{*}{91} & Ведущий программист & 36(40\%) & 1\\
& & Программист 1 & 32(35\%) & 1 \\
& & Программист 2 & 23(25\%) & 1 \\
\hline
\multirow{2}{0.2\textwidth}{Внедрение} & \multirow{3}{*}{19} & Ведущий программист & 8(42\%) & 1\\
& & Программист 2 & 11(58\%) & 1 \\
\hline
\end{tabular}
\end{table}
\normalsize

\begin{table}[ht]\footnotesize
\caption{Календарный ленточный график работ}
\begin{tabular}{|l|l|llll|llll|llll|llll|}
\hline
Стадия разработки & Должность исполнителя & \multicolumn{16}{c|}{Трудоемкость} \\ 
\hline
\multirow{2}{0.2\textwidth}{Техническое задание} & Ведущий программист & \multicolumn{4}{l|}{\cellcolor[gray]{0.6}18} & & & & & & & & & & & &\\
& Программист 1 & \multicolumn{2}{l|}{\cellcolor[gray]{0.8}6} & {} & {} & & & & & & & & & & & & \\
\hline
\multirow{3}{0.2\textwidth}{Эскизный проект} & Ведущий программист & & & & & \multicolumn{4}{l|}{\cellcolor[gray]{0.8}26} & & & & & & & & \\
& Программист 1 & & & & & \multicolumn{4}{l|}{\cellcolor[gray]{0.6}27} & & & & & & & &\\
& Программист 2 & & & & &  \cellcolor[gray]{0.8}7 & {} & {} & \cellcolor[gray]{0.8}7 & & & & & & & & \\
\hline
\multirow{3}{0.2\textwidth}{Технорабочий проект} & Ведущий программист & & & & & & & & &\multicolumn{4}{l|}{\cellcolor[gray]{0.6}36} & & & & \\
& Программист 1 & & & & & & & & &\multicolumn{4}{l|}{\cellcolor[gray]{0.8}32} & & & & \\
& Программист 2 & & & & & & & & & {} & \multicolumn{3}{l|}{\cellcolor[gray]{0.8}23} & {} & & & \\
\hline
\multirow{2}{0.2\textwidth}{Внедрение} & Ведущий программист & & & & & & & & & & & & &\multicolumn{4}{l|}{\cellcolor[gray]{0.8}9} \\
& Программист 1 & & & & & & & & & & & & & \multicolumn{4}{l|}{\cellcolor[gray]{0.6}10}\\
\hline
\end{tabular}
\label{tab:timeline}
\end{table}

\normalsize 

Из таблицы~\ref{tab:timeline} видно, что благодаря параллельной работе ведущего программиста и программистов можно добиться сокращения сроков разработки с  204 дней до $18 + 27  + 36 + 11 = 92$ дней, т.е. в 2.2 раза.

\section{Расчет затрат на разработку ПП}

Затраты на выполнение проекта сосстоят из затрат на заработную плату исполнителям, затрат на закупку или аренду оборудования, затрат на организацию рабочих мест и затрат на накладные расходы. В таблице~\ref{tab:zarp} приведены затраты на заработную плату и страховые взносы во внебюджетные фонды. Суммарно эти отчисления для органиаций, осуществляющих деятельность в области информационных технологий, составляют 14\%: в Пенсионный фонд (ПФ)~--- 8\%, в Фонд социального стразования (ФСС), Федеральный и территориальный фонды обязательного медицинского страхования (ФФОМС и ТФОМС)~--- по 2\%. Для ведущего программиста предполагается ставка 2500 рублей за полный рабочий день, для первого программиста~--- 1500 рублей, для второго~--- 1000 рублей. 

\begin{table}[ht!]\footnotesize
\caption{Затраты на зарплату и отчисления во внебюджетные фонды}
\begin{tabular}{|l|l|l|l|l|l|l|}
\hline
& \multicolumn{6}{c|}{Февраль}\\
\hline
Исполниель & Рабочих дней & Зарплата & ПФ & ФСС & ФФОМС & ТФОМС\\
\hline
Ведущий программист & 20 & 50000 & 4000 & 1000 & 1000 & 1000\\
\hline
Программист 1 & 20 & 30000 & 2400 & 600 & 600 & 600\\
\hline
Программист 2 & 19 & 19000 & 1520 & 380 & 380 & 380\\
\hline
Итого: & \multicolumn{6}{r|}{112860}\\
\hline
& \multicolumn{6}{c|}{Март}\\
\hline
Исполниель & Рабочих дней & Зарплата & ПФ & ФСС & ФФОМС & ТФОМС\\
\hline
Ведущий программист & 22 & 55000 & 4400 & 1100 & 1100 & 1100\\
\hline
Программист 1 & 22 & 33000 & 2640 & 660 & 660 & 660\\
\hline
Программист 2 & 6 & 6000 & 480 & 120 & 120 & 120\\
\hline
Итого: & \multicolumn{6}{r|}{107160}\\
\hline
& \multicolumn{6}{c|}{Апрель}\\
\hline
Исполниель & Рабочих дней & Зарплата & ПФ & ФСС & ФФОМС & ТФОМС\\
\hline
Ведущий программист & 21 & 52500 & 4200 & 1050 & 1050 & 1050\\
\hline
Программист 1 & 21 & 31500 & 2520 & 630 & 630 & 630\\
\hline
Программист 2 & 10 & 10000 & 800 & 200 & 200 & 200\\
\hline
Итого: & \multicolumn{6}{r|}{107160}\\
\hline
& \multicolumn{6}{c|}{Май}\\
\hline
Исполниель & Рабочих дней & Зарплата & ПФ & ФСС & ФФОМС & ТФОМС\\
\hline
Ведущий программист & 20 & 50000 & 4000 & 1000 & 1000 & 1000\\
\hline
Программист 1 & 20 & 30000 & 2400 & 600 & 600 & 600\\
\hline
Программист 2 & 19 & 19000 & 1520 & 380 & 380 & 380\\
\hline
Итого: & \multicolumn{6}{r|}{112860}\\
\hline
& \multicolumn{6}{c|}{Июнь}\\
\hline
Исполниель & Рабочих дней & Зарплата & ПФ & ФСС & ФФОМС & ТФОМС\\
\hline
Ведущий программист & 16 & 40000 & 3200 & 800 & 800 & 800\\
\hline
Программист 1 & 15 & 22500 & 1800 & 450 & 450 & 450\\
\hline
Программист 2 & 3 & 3000 & 240 & 60 & 60 & 60\\
\hline
Итого: & \multicolumn{6}{r|}{74670}\\
\hline
Общая сумма: & \multicolumn{6}{r|}{446310}\\
\hline
\end{tabular}
\label{tab:zarp}
\end{table}
\normalsize

Расходы на материалы, необходимые для разработки ПП указаны в таблице~\ref{tab:materials}.

\begin{table}[ht!]\footnotesize
\caption{Материальные затраты}
\begin{tabular}{|c|p{0.25\textwidth}|l|c|c|c|}
\hline
\No & Наименование материала & Единица измерения & Кол-во & Цена за единицу, руб. & Сумма, руб.\\
\hline
1 & Бумага А4 & Пачка 500 л. & 1 & 150 & 150 \\
\hline
2 & Картридж для лазерного принтера Brother HL-2030R & Шт. & 2 & 1400 & 2800 \\
\hline
\multicolumn{5}{|r|}{Итого:} & 2950 \\
\hline
\end{tabular}
\label{tab:materials}
\end{table}

При разработке понадобится 3 компьютера (по одному на человека), один принтер, компьютерная мебель (3 комплекта), аксесуары для компьютеров (3 компьютера). Стоимость одной ПЭВМ составляет 35000 рублей. Месячная норма амортизации $K = 4\%$ (Срок службы~--- 25 месяцев). Срок службы составляет 3 года, месячная амортизация $K = 2.78\%$. Срок службы компьютерной мебели: столы~--- 5 лет, стулья~--- 2 года, месячные нормы амортизации~--- $K = 1.67\%$ и $K = 4.17\%$ соответственно. компьютерные аксесуары имеют срок службы 2 года, норма амортизации $K = 4.17\%$.

Затраты на амортизацию приведены в таблице~\ref{tab:amort}

\begin{table}[ht]\footnotesize
\caption{Амортизационные отчисления}
\begin{tabular}{|l|c|c|c|c|c|}
\hline
\parbox{20mm}{Наименование \\ оборудования} & \parbox{20mm}{Балансовая\\цена, руб.} & Кол-во, шт. & К, \% & {Время использования, мес.} & {Сумма отчислений, руб.} \\
\hline
Компьютер & 35000 & 3 & 4 & 4 & 16800 \\
\hline
Принтер & 2600 & 1 & 2.78 & 4 & 290 \\
\hline
Стол & 3000 & 3 & 1.67 & 4 & 600 \\
\hline
Стул & 1000 & 3 & 4.17 & 4 & 500 \\
\hline
Клавиатура & 500 & 3 & 4.17 & 4 & 350 \\
\hline
Мышь & 500 & 3 & 4.17 & 4 & 350 \\
\hline
\multicolumn{5}{|r|}{Итого:} & 18790 \\
\hline
\end{tabular}
\label{tab:amort}
\end{table}
\normalsize

Таке необходимо учесть аренднуюю полату за помещение, которая составляет 14000 рублей за месяц. За 4 месяца на аренду помещения уйдет 56000 рублей.

Общие затраты на разработку ПП составят $446310 + 2950 + 18790 + 56000 = 524050$ рублей.

\section{Расчет экономической эффективности}

Основными показателями экономической эффективности является чистый дисконтированный доход (ЧДД) и сок окупаемости вложенных средств.

Чистый дисконтированный доход определяется по формуле:

\begin{equation}
\text{ЧДД} = \sum_{t=0}^{T}{R_t - \text{З}_t \cdot \frac{1}{(1 + E)^t}}
\label{F:ChDD}
\end{equation}

где $T$~--- горизонт рачета по месяцам; $t$~--- период расчета; $R_t$~--- результат достигнутый на шаге (стоимость)$t$; $\text{З}_t$~--- затраты; $E$~--- приемлимая для инвестора норма прибыли на вложенный капитал.

Коэффициент $E$ установим равным ставке рефинансирования ЦБ РФ~--- 8\% годовых (0.67\% в месяц). В результате анализа рынка программной продукции, аналогичной разрабатваемой, планируется продажа 2 единц в 1-й месяц и 3 единицы в остальные. Планируемая цена ПП составляет 60000 рублей. Предполагаемые накладные расходы, связанные с реализацией составят 2000 рублей в месяц.

В таблице~\ref{tab:ChDD} приведен расчет ЧДД по месяцам работы над проектом, а на Рисунке~\ref{fig:ChDD} изображен график ЧДД (начерчен до момента, когда ЧДД принимает положительные значения).

\begin{table}[ht]\footnotesize
\caption{Расчет ЧДД}
\begin{tabular}{|c|c|c|c|c|c|}
\hline
Месяц & \parbox{20mm}{Текущие затраты, руб.}& \parbox{30mm}{Затраты с начала \\года, руб.}&\parbox{20mm} {Текущий \\доход, руб.}& \parbox{30mm}{Доход с начала\\года, руб.}& ЧДД, руб.\\
\hline
1 & 66451 & 66451 & 0 & 0 &-66451\\
\hline
2 & 126201 & 192652 & 0 & 0 &-192652\\
\hline 
3 & 126201 & 318853 & 0 & 0 &-318853\\
\hline
4 & 131901 & 450754 & 0 & 0 & -450754\\
\hline
5 & 73296 & 524050 & 60000 & 60000 &-464050\\
\hline
6 & 2000 & 526050 & 180000 & 240000 & -286050\\
\hline
7 & 2000 & 528050 & 180000 & 420000 & -108050\\
\hline
7 & 2000 & 530050 & 180000 & 600000 & 69950\\
\hline
\end{tabular}
\label{tab:ChDD}
\end{table}
\normalsize

 Срок окупаемости проекта~--- 8 месяцев.
 
\begin{figure}[ht]
 \centering
 \begin{tikzpicture}
    \begin{axis}[
     width = 11cm,
     xlabel = {Время, мес.},
     ylabel = {ЧДД, руб.},
     xtick align = center,
     yminorgrids, ymajorgrids,
     xmajorgrids,
     minor y tick num = 4,
     legend style={at={(0.74,0.74)}, anchor=southwest}
    ],
    \addplot[black!80!black, mark=o, smooth] plot coordinates {
        (1,-66451)
        (2,-192652)
        (3,-318853)
        (4,-450754)
        (5,-464050)
        (6,-286050)
        (7,-108050)
        (8,69950)
    };
    \end{axis}
 \end{tikzpicture}
 \caption{График изменения ЧДД}
 \label{fig:ChDD}
\end{figure}


\section{Выводы}

По результатам расчета трудоемкости выполнения работ, затрат на выполнение работ и прибыли от реализации программного продукта был найден чистый дисконтированный доход по месяцам разработки и реализации программного продукта. На основе полученных данных можно сделать вывод, что проект эффективен и срок его окупаемости составляет 8 месяцев с начала разработки.
