%% Преамбула TeX-файла

% 1. Стиль и язык
\documentclass[utf8x, 14pt]{G7-32} % Стиль (по умолчанию будет 14pt)

% Остальные стандартные настройки убраны в preamble-std.tex
\include{preamble-std}
\usepackage{pdfpages}
\usepackage{longtable, multirow, rotating, color, colortbl}
\usepackage{amsmath}
\usepackage{tikz}
\usepackage{pgfplots}
\begin{document}

\frontmatter % выключает нумерацию ВСЕГО; здесь начинаются ненумерованные главы: реферат, введение, глоссарий, сокращения и прочее

% Команды \breakingbeforechapters и \nonbreakingbeforechapters
% управляют разрывом страницы перед главами.
% По-умолчанию страница разрывается.

% \nobreakingbeforechapters
% \breakingbeforechapters

\includepdf[pages=-]{inc/doc/titul.pdf}



\tableofcontents


\Introduction

Здесь будет введение. В записке должна быть хотя бы одна ссылка на литературу \cite{sqlite}.


\mainmatter % это включает нумерацию глав и секций в документе ниже


\chapter{Аналитический раздел}
\label{cha:analysis}

В данном разделе проводится обзор принципов функционирования и синтаксиса системы GPSS, а также производится выбор подмножества возможностей GPSS, которые следует реализовать в разрабатываемой системе.

\section{Краткий обзор GPSS}

GPSS стал одним из первых языков моделирования, облегчающих процесс написания имитационных программ. Он был создан в виде конечного продукта Джефри Гордоном в фирме IBM в 1962~г.\cite{ImitGPSS} В свое время он входил в десятку лучших языков программирования и по сей день широко используется для решения практических задач. Наиболее современной версией GPSS для персональных компьютеров на данный момент является пакет GPSS World, разработанный компанией Minuteman Software.

Описание системы на GPSS представляет собой последовательность блоков,
каждый из которых соответствует некоторому оператору (подпрограмме). Каждый блок
имеет определенное количество параметров (полей).


Основой имитационных алгоритмов GPSS является дискретно-событийный подход~--- моделирование системы в дискретные моменты времени, когда происходят события, отражающие последовательность изменения состояний системы во времени.\cite{ImitGPSS}

\section{Объекты языка GPSS}

Основными Объектами языка GPSS являются транзакты и блоки, которые отображают соответственно динамические и статические объекты моделируемой системы.

Транзакты~--- динамические элементы GPSS-модели. В реальной системе транзактам могут соответствовать такие элементы как заявка, покупатель автомобиль и др. Состояние транзакта в процессе моделирования характеризуется следующими атрибутами:

\begin{enumerate}
\item параметры~--- набор значений связанных с транзактом. Каждый транзакт может иметь произвольное число параметров. Каждый параметр имеет уникальный номер, по которому на него можно сослаться;
\item приоритет~--- определяет порядок продвижения транзактов при конкурировании за общий ресурс;
\item текущий блок~--- номер блока, в котором транзакт находится в данный момент;
\item следующий блок~--- номер блока, в который транзакт попытается войти;
\item время появления транзакта~---  момент времени в который транзакт был создан;
\item состояние~--- состояние, показывающее в каких списках транзакт находится в данный момент. Транзакт может находиться в одном из следующих состояний:
    \begin{enumerate}
    \item активен~--- транзакт находится в списке текущих событий и имеет наивысший приоритет;
    \item приостановлен~--- транзакт находится в списке будущих событий либо в списке текущих событий, но с меньшим приоритетом;
    \item пассивен~--- транзакт находится в списке прерываний, списке синхронизации, списке блокировок или списке пользователя;
    \item завершен~--- транзакт уничтожен и больше не участвует в модели.
    \end{enumerate}
    Диаграмма состояний транзакта показана на Рисунке~\ref{fig:transactionState}.
\end{enumerate}


\begin{figure}[ht]
  \centering
  \includegraphics[width=\textwidth]{inc/dia/transactionState}
  \caption{Состояния транзакта}
  \label{fig:transactionState}
\end{figure}


Блоки~--- статические элементы GPSS-модели. Модель в GPSS может быть представлена как диаграмма блоков, т.е. ориентированный граф, узлами которого являются блоки, а дугам~--- направления движения транзактов. с каждым блоком связано некоторое действие, изменяющее состояние прочих элементов модели. Транзакты проходят блоки один за другим, до тех пор пока не достигнут блока TERMINATE. В ряде случаев транзакт может быть остановлен в одном из блоков до наступления некоторого события.


Помимо транзактов и блоков в GPSS используются следующие объекты: устройства, многоканальные устройства (хранилища, памяти), ключи, очереди, списки пользователя и др.


\section{Управления процессом моделирования в GPSS}

В системе GPSS интерпретатор поддерживает сложные структуры организации списков (см. Рисунок~\ref{fig:GPSSChains}).\cite{ImitGPSS} Два основных из них~--- список текущих событий (СТС) и список будущих событий (СБС).

\begin{figure}[ht]
  \centering
  \includegraphics[width=\textwidth]{inc/dia/gpss}
  \caption{Списки GPSS}
  \label{fig:GPSSChains}
\end{figure}

В СТС входят все события запланированные  на текущий  момент модельного времени. Интерпретатор в первую очередь просматривает этот список и перемещает по модели те транзакты, для которых выполнены все условия. Если таких транзактов в списке не оказалось интерпретатор обращается к СБС. Он переносит все события, запланированные на ближайший момент времени и вновь возвращается к просмотру СТС. Перенос также осуществляется в случае совпадения текущего момента времени с моментом наступления ближайшего события из СБС.

В целях эффективной организации просмотра транзактов, движение которых заблокировано (например, из-за занятости некоторого ресурса), используются следующие вспомогательные списки:

\begin{itemize}
\item списки блокировок~--- списки транзактов, которые ожидают освобождения некоторого ресурса;
\item список прерываний~--- содержит транзакты, прерванные во время обслуживания. Используется для организации обслуживания одноканальных устройств с абсолютным приоритетом;
\item списки синхронизации~--- содержат транзакты одного семейства (созданные блоком SPLIT), которые ожидают синхронизации в блоках (MATCH, ASSEMBLE или GATHER);
\item списки пользователя~--- содержат транзакты, выведенные пользователем из СТС с помощью блока LINK. Транзакты могут быть возвращены в СТС помощью блока UNLINK.
\end{itemize}




\section{Выбор подмножества реализуемых блоков}

В современной версии языка GPSS (входящей в пакет GPSS World) поддерживается 53 различных блока.\cite{GPSSRef} В рамках данной работы не представляется возможным реализовать  аналоги каждого из них. Поэтому следует выделить некоторое подмножество блоков, которое с одной не будет слишком обширным, а с другой~--- позволит решать практические или по крайней мере учебные задачи.

В качестве примера рассмотрим задачу из курса Модели оценки качества аппаратно программных комплексов:

\begin{quote}
В вычислительной системе, содержащей N процессоров и M каналов обмена данными, постоянно находятся K задач. Разработать модель, оценивающую производительность системы с учетом отказов и восстановлений процессоров и каналов. Имеется не более L ремонтных бригад, которые ремонтируют отказывающие устройства с бесприоритетной  дисциплиной. Интенсивность отказов, восстановлений, средние времена обработки сообщения и среднее время обдумывания также известны.
\end{quote}

Схема модели данной системы показана на Рисунке~\ref{fig:mainModel}

\begin{figure}[ht]
\centering
\includegraphics[width=\textwidth]{inc/dia/main}
\caption{Схема моделируемой системы}
\label{fig:mainModel}
\end{figure}

Как и подавляющее большинство других задач, данная задача, безусловно, не может быть решена без использования блоков GENERATE, TERMINATE и ADVANCE. Так как моделируемая система является замкнутой, при описании модели не обойтись без блока TRANSFER.

К сожалению, не представляется возможным реализовать процессоры и каналы как многоканальные устройства, т.к. многоканальные устройства в GPSS не поддерживают абсолютные приоритеты и не позволяют смоделировать выход из строя отдельных каналов устройства. Однако, требуемую систему можно описать при помощи множества одноканальных устройств и блока TRANSFER в режиме ALL. Таким образом, также понадобятся блоки SEIZE и RELEASE. Для моделирования отказов устройств можно воспользоваться блоками FAVAIL и FUNAVAIL либо блоками PREEMPT и RETURN.

Наконец, доступные ремонтные бригады можно смоделировать с помощью многоканального устройства. Соответственно, понадобятся блоки ENTER и LEAVE.

Приблизительная модель системы показана в Листинге~\ref{lst:sample01}

\lstinputlisting[caption=Приблизительная модель системы,label=lst:sample01]{inc/src/analysModel.gpss}

Таким образом, разрабатываемая система имитационного моделирования должна поддерживать аналоги по крайней мере следующих блоков: ADVANCE, ENTER, GENERATE, LEAVE, PREEMPT, RELEASE, RETURN, SEIZE, TERMINATE и TRANSFER.

\section{Описание выбранных блоков}

Ниже представлено описание выбранных блоков в соответствии со справочным руководством GPSS World.\cite{GPSSRef}

\subsection*{ADVANCE A,B}

Блок ADVANCE осуществляет задержку продвижения транзактов на заданный промежуток времени.

A~--- Среднее время задержки. Не обязательный параметр. Значение по умолчанию~---~0.

B~--- Максимально допустимое отклонение времени задержки либо функция-модификатор.

\subsection*{ENTER A,B}

При входе в блок ENTER транзакт либо занимает заданное количество каналов указанного многоканального устройства либо блокируется до его освобождения.

A~--- Имя или номер многоканального устройства. Обязательный параметр.

B~--- Число требуемых каналов. Не обязательный параметр. Значение по умолчанию~--- 1.

\subsection*{GENERATE A,B,C,D,E}

Блок GENERATE предназначен для создания новых транзактов.

A~--- Среднее время между генерацией последовательных заявок. Не обязательный параметр.

B~--- Максимальное допустимое отклонение времени генерации либо функция-модификатор. Не обязательный параметр.

С~--- Задержка до начала генерации первого транзакта. Не обязательный параметр.

D~--- Ограничение на максимальное допустимое число созданных транзактов. Не обязательный параметр. Пол умолчанию ограничение отсутствует.

E~--- Уровень приоритета создаваемых заявок. Не обязательный параметр. Значение по умолчанию~--- 0.

\subsection*{LEAVE A,B}

При входе в блок LEAVE транзакт освобождает заданное число каналов указанного многоканального устройства.

A~--- Имя или номер многоканального устройства. Обязательный параметр.

B~--- Число требуемых каналов. Не обязательный параметр. Значение по умолчанию~--- 1.

\subsection*{PREEMPT A,B,C,D,E}

Блок PREEMPT подобен блоку SEIZE и вошедший в него транзакт также пытается занять указанное одноканальное устройство. Однако, данный блок позволяет транзакту занять устройство, даже если в данный момент оно уже занято другим транзактом, при соблюдении ряда условий, определяемых параметрами блока.

A~--- Имя или номер одноканального устройства. Обязательный параметр.

B~--- задает режим работы блока. PR~--- режим приоритетов. По умолчанию~--- режим прерываний. В режиме прерываний транзакт может вытеснить из устройства любой другой транзакт, если тот в свою очередь не захватил устройство через блок PREEMPT. В режиме приоритетов транзакт может вытеснить любой транзакт с меньшим приоритетом.

C~--- задает номер блока, куда будет направлен транзакт вытесненный и устройства в результате действия блока PREEMPT. 

D~--- задает номер параметра вытесненного транзакта, в  котором будет сохранено время, которое осталось транзакту до окончания обработки а устройстве.

E~--- задает режим удаления вытесненного транзакта. RE~--- вытесненное сообщение удаляется из устройства и более не претендует на владение им. Требует обязательного указания параметра C. Значение по умолчанию~--- вытесненный транзакт будет вновь пытаться занять устройство. 

\subsection*{RELEASE A}

Блок RELEASE освобождает одноканальное устройство.

A~--- Имя или номер одноканального устройства. Обязательный параметр.

\subsection*{RETURN A}

Блок RELEASE освобождает одноканальное устройство.

A~--- Имя или номер одноканального устройства. Обязательный параметр.

\subsection*{SEIZE A}

При входе в блок SEIZE транзакт занимает указанное одноканальное устройство либо блокируется до его освобождения.

A~--- Имя или номер одноканального устройства. Обязательный параметр.

\subsection*{TERMINATE A}

Блок TERMINATE завершает поступивший в него транзакт. И опционально уменьшает счетчик завершенных транзактов. Когда счетчик достигает нуля имитация останавливается.

A~--- Значение, на которое следует уменьшить счетчик завершенных транзактов. Не обязательный параметр. Значение по умолчанию~--- 0.

\subsection*{TRANSFER A,B,C,D}

Блок TRANSFER является основным средством позволяющим изменить маршрут транзакта и перенаправить его к произвольному блоку модели. Параметр A определяет режим работы блока. Смысл остальных параметров меняется в зависимости от выбранного режима.

\paragraph{Безусловный режим}. Если параметр А пропущен, то блок TRANSFER работает в безусловном режиме. Входящий в блок TRANSFER транзакт переходит к блоку, указанному в поле В.

\paragraph{Статистический режим}. Параметр А является числом от 0 до 1, показывающим какая доля транзактов перейдет к блоку, указанному в параметре С. Остальные транзакты переходят к блоку,ь указанному в параметре B.

\paragraph{Режим BOTH}. Если параметр А равен BOTH, то блок TRANSFER работает в одноименном режиме.В этом режиме каждый входящий транзакт сначала пытается перейти к блоку, указанному в поле В. Если это сделать не удается, транзакт пытается перейти к блоку, указанному в поле С. Если транзакт не сможет перейти ни к тому, ни к другому блоку, он остается в блоке TRANSFER и будет повторять в том порядке попытки перехода при каждом просмотре списка текущих событий, до тех пор, пока не сможет выйти из блока TRANSFER.

\paragraph{Режим ALL}. Если параметр А равен ALL, то блок TRANSFER работает в одноименном режиме.В этом режиме каждый входящий транзакт1 прежде всего пытается перейти к блоку, указанному в поле В. Если транзакт не может войти в этот блок, то последовательно проверяются все блоки в определенном ряду в поисках первого, способного принять это сообщение, включая последний блок, указанный операндом С. Номер каждого проверяемого блока вычисляется как сумма номера предыдущего блока и шага, заданного операндом D. По умолчанию значение операнда D принимается равным 1.

\paragraph{Режим PICK}. Если параметр А равен PICK, то блок TRANSFER работает в одноименном режиме. Этот режим подобен режиму ALL, за тем исключением, что блок назначения выбирается случайным образом с одинаковой вероятностью.

\section{Выводы}

Был проведен обзор устройства системы GPSS и осуществлен выбор подмножества блоков, необходимых для моделирования не сложных систем массового обслуживания. Представлено описание назначения и параметров каждого из выбранных блоков.



\chapter{Конструкторский раздел}

В данном разделе проводится выбор синтаксиса описания моделей в разрабатываемой системе, а также описываются алгоритмы и структуры данных, используемые при формировании моделей и непосредственно при моделировании. 



\section{}

Синтаксис разрабатываемой системы должен быть, на сколько это возможно, схож с синтаксисом системы GPSS. 

Программа на языке GPSS представляет из себя последовательность операторов, каждый из которы описывает тот или иной элемент модели (функцию, блок, устройство и др.). Этот подход естественен для императивных языков программирования, в которых программа является последовательностю комманд, меняющих состояние программы. Однако Haskell относится к категории функциональных языков, программы на которых лписываются как функции, значение которых вычисляется. При этом нет фиксированной, заданной программистом, последовательности операций, которые должны быть выполнены для достижения результата. 

Тем не менее, в языке Haskell предусмотрен механизм, позволяющий описать конкретную последовательность вычислений~--- монады. В сочетании с так называемой do-нотацией, этот механизм позволит проводить описание моделей на Haskel, используя синтаксис схожий с GPSS.

\section{Монады}

Понятие монады в языке Haskell основано на теории категорий. В рамках данной теории монадо может быть определена (не вполне строго) как моноид в категории эндофункторов. Однако для практического использования этого понятия в рамках языка Haskell можно обойтись менее формальным определением.

В соответствии с \cite{Haskell} монада~--- это контейнейрный тип данных (то есть такой, который содержит в себе значения других типов), представляющий собой экземпляр класса Monad определенного в модуле Prelude. 

Под классом в Haskell, понимается не тип данных, как в объектно-оринтированных языках, а набор методов (функций), которые применимы для работы с теми или иными типами данных, для которых объявлены экземпляры заданных классов. Наиболее близким аналогом классам в Haskell являются интерфейсы в таких языках как Java или C\#. Более точно их следует называть классами типов, но т.к. в данной работе используется исключительно функциональная парадигма, в дальнейшем для краткости они будут называться просто классами.

Значения монад можно воспринимать, как значения м неккоторым дополнительным контекстом. В случае монады Maybe значения обладают дополнительным контекстом того, что вычисления могли закончиться неуспешно. Монада IO добавляет контекст, указывающий что получение значений связано с действиями ввода/вывода и потому не является детерминированным и может иметь побочные эффекты. В случае списков (которые также являются монадой) контекстом является то, что значение может являться множественным или отсутствовать.

Класс Monad определен в модуле Prelude следующим образом:

\lstinputlisting[caption=Класс Monad,label=lst:monad]{inc/src/Monad.hs}

Функция \Code{return}\footnote{Следует отметить, что название return никак нельзя назвать удачным, так как оно неизбежно вызывает ассоциации с одноименным оператором из многих императивных языков программирования, на которые она не похожа ничем кроме названия. Данная функция не завершает выполнение функции, а лишь оборачивает переданное значение в монаду.} преобразует переданное ей значение типа a  в монадическое значение типа m a. Другими словами она помещает значение в некоторый контекст по умолчанию, в зависимости от выбранной монады. Для списка это будет списко из одного элемента, для монады IO~--- действие ввода вывода, всегда возвращающее заданное значение и не имеющее побочных эффектов и т.д.

Функция \Code{$>>=$} определяет операцию связывания. Она принимает монадическое значение и передает его функции, которая принимает обычное значение и возвращает монадическое. При этом сохраняется накопленный контекст и к нему добавляется новый, полученный в результате выполнения функции.

Функция \Code{$>>$} также предназначена для связывания и используется в тех случаях, когда переданное монадическое значение не представляет интереса, а значение имеет только переданный с ним контекст вычислений. Для этой функции в классе определена реализация по умолчанию, по этому в большенстве случаев при определении экземпляра класса Monad в явном виде ее не реализуют.

Функция \Code{fail} никогда не вызывается программистом явным образом и предназначена для обработки неуспешного окончания вычислений при сопоставлении с образцом в do-нотации, что позволяет избежать аварийного завершения программы и вернуть неудачу в контексте текущец монады.

\section{Нотация do}

Так как монады находят крайне широкое применение в программах на языке Haskell (в первую очередь, без использования монады IO невозможно осуществить ввод/вывод), в синтаксис языка было добавлено специальное ключевое слово \Code{do}, призванное упростить написание монадических функций, сделать их более читаемыми и избавить от излишнего <<синтаксического мусора>>.

Если в коде программы встречается контрукция с ключевым словом \Code{do}, то транслятор выполняет следующие преобразования\footnote{В приведенных преобразованиях используются управляющие символы \Code{;}, \Code{\{} и \Code{\}}, хотя в реальных программах на языке Haskell их можно встретить довольно редко. Это связано с тем, что в Haskell используется так называемый <<двумерный синтаксис>>: при правильной расстановке отступов, транслятор самостоятельно расставляет точки с запятой и фигурные скобки и в большинтве случаев нет смысла заграмождать ими исходный код. Тем не менее в случае необходимости их можно расставить и явным образом.}:

\begin{verbatim}
1. do {e} → e
2. do {e; es} → e >> do {es}
3. do {let decls; es} → let decls in do {es}
4. do {p <- e; es} → let ok p = do {es}
                          ok _ = fail "..."
                      in e >>= ok
\end{verbatim}

При помощи нотации \Code{do} приведенный ниже фрагмент кода

\begin{verbatim}
foo :: Maybe String
foo = Just 3 >>= (\x -> Just "!" >>= (\y -> Just (show x ++ y)))}
\end{verbatim}

может быть записан в следующей более читаемой форме:

\begin{verbatim}
foo :: Maybe String
foo = do x <- Just 3 
         y <- Just "!" 
         return (show x ++ y)
\end{verbatim}

\section{Монада State}

Часто в процессе вычислений возникает необходимость хранить и изменять некоторое состояние, в завсимости от которого результат вычислений может меняться. Haskell является чистым функциональным языком программирования функции должны быть детерменированы и не иметь побочных эффектов, поэтому текущее состояние обычно передается в функции как еще один параметр, а возвращает функция пару из собственно результата и обновленного состояния.

Для того, чтобы упростить написание функций оперирующих некоторым соостоянием в Haskell была введена монада \Code{State}. Она определена в модуле \Code{Control.Monad.State} следующим образом:

\begin{verbatim}

newtype State s a = State {runState :: s -> (a, s)}

instance Monad (State s) where
    return x = State $ \s -> (x,s)
    (State h) >>= f = State $ \s -> let (a, newState) = h s
                                        (State g) = f a
                                    in g newState

\end{verbatim}

Функция \Code{return} создает вычисление с состоянием, которое всегда возвращает один и тот же результат и оставляет переданное в него состояние без изменений. Функция \Code{$>>=$} <<склеивает>> два вычисления с состоянием так, что конечное  состояние первого становится начальным для второго, а результат и конечное состояние второго вычисления становятся также результатом и конечным состоянием итогового, составного вычисления.

Помимо этого для работы с монадой \Code{State} используются две вспомогательные функции \Code{put} и \Code{get}. Функция \Code{put} является вычислением, которое устанавливает состояние в заданное значение не зависимо от его предыдущего значения и не возвращает никакого результата (точнее возвращает кортеж нулевой длины \Code{()}). Функция \Code{get} возвращает текущее состояние и оставляет его без изменений. 

\section{Описание модели как вычислене с состоянием}

Описание модели на языке GPSS представляет из себя последовательность блоков. В Haskell такое описание удобно представить как последовательность функций, каждая из которых добавляет к уже сформированной модели очередной блок. 

Такой процесс удобно представить как вычисление с состоянием. Каждая функция, формирующая блок, помимо параметров самого блока должна принимать текущее состояние~--- список уже сформированных к данному моменту боков в пордке их формирования. В качестве результата функция возвращает новое состояние~--- модель к которой добавлен только что сформированный блок (см. Рисунок \ref{fig:StateIDEF}).

\begin{figure}[ht]
  \centering
  \includegraphics[width=\textwidth]{inc/dia/StateIDEF}
  \caption{Процесс формирования модели}
  \label{fig:StateIDEF}
\end{figure}

Для реализации такого механизма целесообразно воспользоваться монадой \Code{State}, что позволит скрыть явную передачу состояния от одной функции к другой. А использование нотации \Code{do} сделает описание модели почти идентичным синтаксису GPSS:

\begin{verbatim}

model = 
    do generate (10,2)
       advance 3
       terminate 1

\end{verbatim}


\section{Функции формирования блоков}

В языке GPSS имена всех блоков пишутся с заглавной буквы. Параметры отделяются от имени блока пробелом и разделяются запятыми. Синтаксис Haskell не позволяет в точности повторить эти соглашения. Имена функций в Haskell обязаны начинаться со строчной буквы. Параметры функций обычно разделяются пробелами и не берутся в скобки (каррированные функции) либо заключаются в скобки и разделяются запятыми (не каррированные функции)\footnote{Строго говоря все функции в Haskell принимают ровно один параметр. Функции от N парааметров на самом деле принимают один параметр и возвращают функцию от N-1 параметра (каррированные функции) либо принимают параметр-кортеж (некаррированные).}. Оба варианта описания параметров одинакоово близки к синтаксису GPSS и можно выбрать любой из них, однако для второго варианта существенно легче реализовать перегрузку функций.

\section{Перегрузка функций}


\chapter{Технологический раздел}

В данном разделе описывается реализация разработанной системы и демонстрационной программы, средства разработки, сборки и развертывания, а также тестирование.

\section{Выбор средств программной реализации}

Так как целью работы является разработка библиотеки для языка Haskell, целесообразно вести разработку на этом же языке. Язык Haskell относится к функциональным языкам общего назначения и обладает следующими особенностями:

\begin{itemize}
\item Чистые функции. Большинство функций в языке Haskell являются чистыми, то есть детерминированными и не имеющими побочных эффектов. При использовании таких функций программист может быть уверен, что при вызове такой функции не будет произведено каких-либо неявных действий (запись данных в файл, изменение значения переданных параметров, изменение состояния некоторого объекта и т.п.) и на одних и тех же входных параметрах функция всегда вернет одинаковый результат. Это существенно упрощает разработку и тестирование программ.

\item Ленивые вычисления. Все вычисления в Haskell по умолчанию являются ленивыми, то есть ни одно значение не будет вычислено до тех пор, пока его значение действительно не понадобится. Этот механизм позволяет экономить вычислительные мощности и работать со структурами вроде бесконечных списков. Однако при неаккуратном использовании это может привести к неэффективному расходованию памяти.

\item Строгая статическая типизация. Строгая типизация позволяет писать более надежные программы, так как несоответствие типов приведет к сообщению об ошибке, а не к некорректному поведению программы и будет быстрее обнаружено и исправлено. Статическая типизация позволяет выявить такие ошибки еще на этапе компиляции программы.

\item Автоматическое управление памятью. Как и большинство современных языков программирования Haskell берет на себя выделение и освобождение памяти. Это гарантировать отсутствие в разрабатываемой программе таких ошибок как переполнение буфера, не инициализированные переменные и т.п.

\end{itemize}

Помимо этого для Haskell существует обширный централизованный архив библиотек Hackage\cite{hackage}, и поисковая система Hoogle\cite{hoogle}, позволяющая найти описание функции не только по ее названию но и по сигнатуре.

\subsection{Построение графиков}

Для построения графиков была выбрана система Gnuplot\cite{gnuplot}. Это свободно распространяемая, кросплатформенная программа предназначенная для построения двух- и трехмерных графиков функций, заданных как аналитически, так и в виде наборов данных. Gnuplot поддерживает вывод результатов в различных форматах: растровых (PNG, JPEG), векторных (SVG, PDF), в виде кода LaTeX, в интерактивном режиме и др. Система используется для построения графиков в таких математических пакетах как GNU Octave, Maxima и других.

Для использования возможностей Gnuplot в программах на Haskell существует несколько библиотек, опубликованных на Hackage. В данной работе была использована библиотека EasyPlot.


\subsection{Построение пользовательского интерфейса}

Для построения пользовательского интерфейса демонстрационной программы была использована библиотека wxHaskell\cite{wxhaskell}, которая в свою очередь является оберткой вокруг библиотеки для построения пользовательского интерфейса на C++ wxWidgets. 

Особенности wxWidgets:

\begin{itemize}
\item Созданные с помощью данной библиотеки приложения переносимы на большинство современных ОС.

\item В разработанном интерфейсе используются элементы управления привычные для пользователей целевой ОС. То есть стиль интерфейса программы будет отличаться на различных ОС и будет соответствовать рекомендуемому стилю для конкретной системы.

\end{itemize}

wxHaskell является надстройкой над wxWidgets, позволяющей создавать графический интерфейс к программам на языке Haskell. Она поддерживает большую часть функционала wxWidgets, позволяет описывать интерфейс в <<декларативном>> стиле с использованием функциональных связок и абстракций высокого уровня. Библиотека особенно удобна для создания демонстрационных версий программ, так как во многом берет на себя решение задачи корректного расположения элементов управления на экране. 

\subsection{Сборка и развертывание библиотеки}

Для сборки разработанной библиотеки и развертывания ее на целевой машине была использована система Cabal. Данная система предоставляет единый интерфейс для создания и установки инсталяционных пакетов с программами и библиотеками на Haskell. Система связана с архивом библиотек Hackage и позволяет устанавливать хранящиеся там пакеты и оформить собственную программу в виде пакета для Hackage.

Информация о создаваемом пакете указывается в файле \Code{.cabal} в директории проекта. В нем указываются:
    
\begin{itemize}
\item имя пакета;

\item текущая версия;

\item информация об авторе и лицензии;

\item допустимые версии компилятора;

\item используемые расширения компилятора;

\item входящие в состав пакета библиотеки и исполняемые программы;

\item пакеты Hackage, необходимые для работы пакета;

\item и др.

\end{itemize}
    
    
Фрагмент файла \Code{.cabal} для разработанной библиотеки приведен ниже.

\lstinputlisting[caption=Фрагмент описания пакета,label=lst:cabal]{inc/src/.cabal}

Сборка и установка пакета выполняется следующими командами:

\begin{itemize}
\item \Code{cabal configure}~--- подготовка к сборке программы: определение целевой платформы, зависимостей и др.;

\item \Code{cabal build}~--- запуск процесса компиляции;

\item \Code{cabal install}~--- установка пакета в систему. Включает в себя первые две команды;

\item \Code{cabal clean}~--- удаляет все временные файлы созданные предыдущими командами;

\end{itemize}
    
    

%\section{Демонстрационная программа}

%Для демонстрации возможностей разработанной библиотеки, а также в целях дополнительного тестирования была разработана демонстрационная программа. В программе производится расчет производительности системы массового обслуживания при заданных параметрах и варьировании одного из них. Расчет производительности ведется как путем имитационного моделирования при помощи разработанной библиотеки, так и с помощью аналитической модели. Программа выводит на экран графики, позволяющие сравнить значения полученные путем моделирования с полученными теоретически. Структура разработанной программы показа на рисунке~\ref{fig:demoStruct}



\section{Тестирование}

Для модульного тестирования разработанной библиотеки использовалась библиотека HUnit\cite{HUnit}. Это фреймворк основанный на идеях JUnit, но предназначенный для тестирования программ на Haskell.

Типичный базовый тест состоит из текстового описания, вычисляемого выражения и ожидаемого результата. Базовые тесты объединяются в группы, те, в свою очередь, могут объединяться в большие группы и т.д. В итоге образуется единая древовидная структура тестов. Ниже показан пример описания теста.

\begin{verbatim}
emptyFEC = TestCase (assertEqual "for (addFE [] (1,defTransact))," 
                                 ([(1,defTransact)]) 
                                 (addFE [] (1,defTransact))
                    )
\end{verbatim}

По результатам выполнения тестов HUnit выводит следующую статистику: общее число тестов, число проведенных тестов, число тестов вызвавших непредвиденное исключение (что говорит об ошибке в самом тесте) и число тестов, закончившихся неудачей (что обычно говорит об ошибке в тестируемой программе).

В таблице~\ref{tab:chainsTest} приведен протокол тестирования функции \Code{addFE}, добавляющей новое событие в список будущих событий.


\begin{table}[h!]
\caption{}
\label{tab:chainsTest}
\begin{tabular}{|l|p{0.7\textwidth}|}
\hline
Название теста & emptyFEC\\
\hline
Описание теста & Добавление события в пустой список\\
\hline
Ожидаемый результат & Список из одного события\\
\hline
Результат & Тест пройден\\
\hline
\hline
Название теста & nearestAddFE\\
\hline
Описание теста & Добавление в список события с наименьшим временем наступления\\
\hline
Ожидаемый результат & Добавленное событие становится первым в списке\\
\hline
Результат & Тест пройден\\
\hline
\hline
Название теста & lastAddFE\\
\hline
Описание теста & Добавление в список события с наибольшим временем наступления\\
\hline
Ожидаемый результат & Добавленное событие становится последним в списке\\
\hline
Результат & Тест пройден\\
\hline
\hline
Название теста & middleAddFE\\
\hline
Описание теста & Добавление в список события с промежуточным временем наступления\\
\hline
Ожидаемый результат & Добавленное событие занимает место в списке в соответствии со своим временем наступления\\
\hline
Результат & Тест пройден\\
\hline
\hline
Название теста & multyAddFE\\
\hline
Описание теста & Добавление в список события с тем же  временем наступления, что и у  одного из событий в списке\\
\hline
Ожидаемый результат & Добавленное событие занимает место в списке сразу за событием с тем же временем наступления\\
\hline
Результат & Тест пройден \\
\hline
\end{tabular}
\end{table}

В общей сложности было проведено 148 тестов. Степень покрытия кода тестами показана в таблице~\ref{tab:coverage}. Вывод тестовой программы представлен ниже.

\begin{verbatim}
Cases: 148  Tried: 148  Errors: 0  Failures: 0
\end{verbatim}

\begin{table}[ht!]
\caption{Степень покрытия кода тестами}
\begin{tabular}{|p{0.2\textwidth}|c|c|c|}
\hline
Подсистема & Покрытие функций & Покрытие условий & Покрытие выражений\\
\hline
\parbox{0.25\textwidth} {Формирование \\моделей }& 82\% & 100\% & 83\% \\
\hline
\parbox{0.25\textwidth} {Имитационное \\моделирование }& 72\% & 96\% & 93\% \\
\hline
\end{tabular}
\label{tab:coverage}
\end{table}

Тестирование проводилось на машинах с различными операционными системами: Ubuntu 12.04, Ubuntu 13.04, MS Windows 7. На всех библиотека скомпилировалась без ошибок и все тесты были пройдены успешно.


\subsection{Тестирование стохастических функций}

При организации имитационного моделирования важную роль играют функции генерации случайных величин с заданным законом распределения. При тестировании таких функций приходится отказаться от описанного выше принципа <<выражение --- ожидаемое значение>>, поскольку результат вычисления функции различается от запуска к запуску.

Для тестирования таких функций следует многократно вычислить значение тестируемой функции и сравнить собранную статистику с ожидаемым распределением. В данной работе в качестве критерия соответствия полученных значений ожидаемому распределению использовался критерий согласия Пирсона.

Для вычисления статистики по этому критерию и вычисления квантилей распределения $\chi^2$ использовался пакет statistics из архива библиотек Hackage.

\subsection{Сравнение аналитической и имитационной модели}

В качестве дополнительной проверки на корректность разработанных алгоритмов и безошибочность их реализации был проведен ряд опытов с демонстрационной программой. Были построены графики зависимости производительности моделируемой системы от различных параметров при прочих фиксированных параметрах.

\subsection*{Опыт 1}

Варьируется число задач в системе. Количество процессоров и каналов равно единицы. Интенсивность обдумывания равна единице. Отказов и восстановлений нет. Интенсивности обработки на канальной и процессорной фазе равны 5. Результаты показаны на рисунке~\ref{fig:plot1}.

\begin{figure}[ht]
  \centering
  \includegraphics[width=\textwidth]{inc/pdf/plot1}
  \caption{Опыт 1}
  \label{fig:plot1}
\end{figure}

\subsection*{Опыт2}

Опыт проводится при тех же параметрах, но число каналов и интенсивность обслуживания на процессорной фазе увеличены в двое. Результаты показаны на~\ref{fig:plot1a}.

\begin{figure}[ht]
  \centering
  \includegraphics[width=\textwidth]{inc/pdf/plot1a}
  \caption{Опыт 2}
  \label{fig:plot1a}
\end{figure}


\subsection*{Опыт3}

Варьируется число процессоров. Число задач и каналов равняется 10. Интенсивность обдумывания равна единице. Интенсивности обработки задач на процессорной и канальной фазах равны 5. Результаты опыта показаны на рисунке~\ref{fig:plot2}.

\begin{figure}[ht!]
  \centering
  \includegraphics[width=\textwidth]{inc/pdf/plot2}
  \caption{Опыт 3}
  \label{fig:plot2}
\end{figure}

\subsection*{Опыт4}

Опыт проводится при тех же параметрах, что и предыдущий, но количество каналов уменьшено вдвое. Результаты опыта показаны на рисунке~\ref{fig:plot2a}.

\begin{figure}[ht!]
  \centering
  \includegraphics[width=\textwidth]{inc/pdf/plot2a}
  \caption{Опыт 4}
  \label{fig:plot2a}
\end{figure}

\subsection*{Опыт5}

Варьируется интенсивность обработки на канальной фазе. количество задач равно 10. Количество процессоров и каналов равно 4. Интенсивность обдумывания~--- 5, интенсивность обработки на процессорной фазе~--- 20 Результаты опыта показаны на рисунке~\ref{fig:plot3}.

\begin{figure}[ht!]
  \centering
  \includegraphics[width=\textwidth]{inc/pdf/plot3}
  \caption{Опыт 5}
  \label{fig:plot3}
\end{figure}

\subsection*{Опыт6}

Опыт проводится при тех же параметрах, что и предыдущий, но количество задач уменьшено вдвое. Результаты опыта показаны на рисунке~\ref{fig:plot3a}.

\begin{figure}[ht!]
  \centering
  \includegraphics[width=\textwidth]{inc/pdf/plot3a}
  \caption{Опыт 6}
  \label{fig:plot3a}
\end{figure}

\subsection*{Опыт7}

Варьируется интенсивность отказов процессоров. Количество задач равно 10. Количество процессоров и каналов равно 4. Интенсивность обдумывания~--- 1, интенсивность обработки на процессорной  и канальной фазах~--- 3. Интенсивность восстановления процессоров равна 10. Количество ремонтных бригад~--- 5.Результаты опыта показаны на рисунке~\ref{fig:plot4}.


\begin{figure}[ht!]
  \centering
  \includegraphics[width=\textwidth]{inc/pdf/plot4}
  \caption{Опыт 7}
  \label{fig:plot4}
\end{figure}

\subsection*{Опыт8}

Опыт проводится при тех же параметрах, что и предыдущий, но интенсивность восстановления процессоров уменьшена вдвое. Результаты опыта показаны на рисунке~\ref{fig:plot4a}.

\begin{figure}[ht!]
  \centering
  \includegraphics[width=\textwidth]{inc/pdf/plot4a}
  \caption{Опыт 8}
  \label{fig:plot4a}
\end{figure}

\subsection*{Опыт9}

Варьируется количество ремонтных бригад. Количество задач и процессоров~--- 5. Количество каналов 15. Интенсивность обдумывания~---1. Интенсивность обработки на процессорной и канальной фазах~---5. Интенсивность отказов каналов~---14, интенсивность восстановлений~---7. Результаты опыта показаны на рисунке~\ref{fig:plot5}.

\begin{figure}[ht!]
  \centering
  \includegraphics[width=\textwidth]{inc/pdf/plot5}
  \caption{Опыт 9}
  \label{fig:plot5}
\end{figure}

\subsection*{Опыт10}

Опыт проводится при тех же параметрах, что и предыдущий, но интенсивность восстановления взята равной интенсивности отказов. Результаты опыта показаны на рисунке~\ref{fig:plot5a}.

\begin{figure}[ht!]
  \centering
  \includegraphics[width=\textwidth]{inc/pdf/plot5a}
  \caption{Опыт 10}
  \label{fig:plot5a}
\end{figure}

Из проведенных опытов видно, что данные полученные при помощи разработанной библиотеки и имитационной модели качественно соответствуют результатам аналитических исследований. Некоторый разброс значений объясняется стохастическим характером модели.

\section{Выводы}

Были реализованы разработанные в предыдущем разделе методы и алгоритмы. Разработанная библиотека была протестирована при помощи модульного тестирования. Также был проведен ряд опытов подтверждающих адекватность построенной имитационной модели и корректность реализованных алгоритмов.


\include{50-economics}

\chapter{Раздел по охране труда}

\section{Гигиеничекие требования к персональным ЭВМ и организации работы}

Разработка ПО требует длительного взаимодействия с вычмслительными системами. Работа с ПЭВМ связана с рядом вредных и опасных факторов, таких как статическое электричество, рентгеновское излучение, электромагнитные поля, блики отраженный свет, ультрафиолетовое излучение. При длительном воздействии на организм эти факторы негативно влияют на здоровье человека.

\subsection{Микроклимат}

Работа как программиста, так и пользователя относится к категории 1а, поскольку не предполагает больших физических усилий. Нормы, установленные СанПиН 2.2.2/2.4.1340-03 для категории работ 1а приведены в таблице~\ref{tab:microclimate}.

\begin{table}[ht]
\caption{Нормы микроклимата}
\begin{tabular}{|l|c|c|c|c|c|c|}
\hline
\multirow{2}{*}{Период год} & \multicolumn{2}{l|}{Температура, $^\circ \mbox{C}$} & \multicolumn{2}{l|}{Влажность, \%} & \multicolumn{2}{l|}{Скорость воздуха, м/с} \\
\cline{2-7}
&Оптим.&Допуст.&Оптим.&Допуст.&Оптим.&Допуст.\\
\hline
Холодный &22--24&21--25&40--60&75&0.1&0.1\\
\hline
Теплый &23--25&22--28&40--60&55 при 28$^\circ \mbox{C}$&0.1&0.1\\
\hline 
\end{tabular}
\label{tab:microclimate}
\end{table}

Вредным фактором при работе с ЭВМ является также запыленность помещения. Этот фактор усугубляется влиянием на частицы пыли электростатических полей персональных компьютеров.

Для устранения несоответствия параметров указанным нормам проектом предусмотренно использование системы кондиционирования как наиболее эффективного и автоматически функционирующего средства.

Нормы установленные содержания в воздухе положительных и отрицательной ионов, установленные СанПиН 2.2.4.1294--03, приведены в таблице~\ref{tab:ions}.

\begin{table}[ht]
\caption{Уровни ионизации воздуха при работе на ПЭВМ}
\begin{tabular}{|l|c|c|}
\hline
\multirow{2}{*}{Уровни} & \multicolumn{2}{l|}{Число ионов в кубометре воздуха}\\
\cline{2-3}
&$n^+$&$n^-$\\
\hline
Минимально необходимое & 400 & 600 \\
\hline
Оптимальное & 1500--3000 & 3000--5000 \\
\hline
Максимально допустимое & 50000 & 50000 \\
\hline
\end{tabular}
\label{tab:ions}
\end{table}

Для обеспечения требуемых уровней предусмотренно использование системы ионизации Сапфир-4А.

\subsection{Шум и вибрации}

Уровень шума на рабочем месте программиста не должен превышать 50 дБА, а уровень вирации не должен превышать норм становленных СанПиН 2.2.2.542--96 (см. таблицу~\ref{tab:vibro}).

\begin{table}[ht]
\caption{Допустимые нормы вибрации на раочих местах с ПЭВМ}
\begin{tabular}{|c|c|c|}
\hline
\parbox{0.4\textwidth}{ Среднегеометрические частоты\\октавных полос, Гц}& \multicolumn{2}{l|}{Допустимые значения по виброскорости}\\
\cline{2-3}
&м/c &дБ\\
\hline
2  & $4.5\times10$ & 79 \\
\hline
4  & $2.2\times10$ & 73 \\
\hline
8  & $1.1\times10$ & 67 \\
\hline
16  & $1.1\times10$ & 67 \\
\hline
31.5 & $1.1\times10$ & 67 \\
\hline
63  & $1.1\times10$ & 67 \\
\hline
\parbox{0.4\textwidth}{ Корректированные значения\\и их уровни в дБ}& $2.0\times10$ & 72\\
\hline
\end{tabular}
\label{tab:vibro}
\end{table}

При разработке ПО внутренними источниками шума являются вентиляторы, а также принтеры и другие перефферийные устройства ЭВМ. Внешние источники шума~--- прежде всего, шум с улицы и из соседних помещений. Постоянные внешние источники шума, превышающего нормы, отсутствуют.

Для устранения превышения нормы проектом предусмотрено применение звукопоглощающих материалов для облицовки стен и потолка помещения, в котором осуществляется работа с вычислительной техникой.

\subsection{Освещение}

Наиболее важным условием эффективной раоты программитов и пользователей является соблюдение оптимальных параметров системы освещения в рабочих помещениях.

Естественное освещение осуществляется через светопроемы, ориентированные в основном на север и северо-восток (для исключения попадания прямых солнечных лучей на экраны компьютеров) и обеспечивает коэффициент естественной освещенности (КЕО) не ниже 1.5\%.

В качестве искуственного освещения проектом предусмотрено использование системы общего освещения. в соответствии с СанПин 2.2.2/2.4.1340--03 освещенность на поверхнности рабочего стола должна находиться в пределах 300--500 лк. Разрешается использование светильников местного освещения для работы в документами (при этом светильники не должны создавать блики на поверхности экрана).

Правильное расположение рабочих мест относительно источников освещения, отсутствие зеркальных поверхностей и использование матовых материалов ограничивает прямую (от источников освещения) и отраженную (от рабочих поверхностей) блескость. При  этом яркость светящихся пооверхностей не привышает $200 \frac{\text{кд}}{\text{м}^2}$, яркость бликов на экране ПЭВМ не превышает $40 \frac{\text{кд}}{\text{м}^2}$, и яркость потолка не превышает $200 \frac{\text{кд}}{\text{м}^2}$.

В соответствии с СанПинН 2.2.2/2.4.1340--03 проектом предусмотрено использование люминисцентных ламп типа ЛБ в качестве источников света при искуственном освещении. В светильниках допускается применение ламп накаливания. Применение газоразрядных ламп в светильниках общего и местного освещения обеспечивает коэффициент пульсации не более 5\%.

Таким образом, проектом обеспечиваются оптимальные условия освещения рабочего помещения.

\subsection{Рентгеновское излучение}

В соответствии с СанПиН 2.2.2/2.4.1340-03 проектом предусмотрено использование ПЭВМ, конструкция которого обеспечивает мощность экспозиционной дозы рентгеновского излучения в любой точке на расстоянии 0.5 м. от экрана и корпуса не более 0.1 мбэр/час (100 мкР/час). Результаты сравнения норм излучения приведены в таблице~\ref{tab:rentgen}.

\begin{table}[ht]
\caption{Сравнение норм рентгеновского излучения в различных стандартах}
\begin{tabular}{|l|c|}
\hline
& Допустимое значение мкР/час, не более \\
\hline
СанПиН 2.2.2/2.4.1340-03 & 100 \\
\hline
ТСО-99 & 500 \\
\hline
MPR II & 500\\
\hline
\end{tabular}
\label{tab:rentgen}
\end{table}

Как видно из таблицы, стандарты MPR II и ТСО--99 предъявляют менее жесткие требования к рентгеновскому излучению, чем СанПиН. Но при соблюдении оптимального расстояния между пользователем и монитором дозы рентгеновского излучения не опасны для большнства людей.

\subsection{Неионизирующие жлектромагнитные излучения}

Допустимые значения параметров неионизирующих излучений в соответствии с СанПин 2.2.2/2.4.1340-03 приведены в таблицах~\ref{tab:U} и~\ref{tab:ro}.

\begin{table}[ht]
\caption{Предельно допустимые значения напряженности электрического поля}
\begin{tabular}{|c|c|}
\hline
Диапазон частот& Допустимые значения \\
\hline
5 Гц -- 2 кГц & 25 В/м \\
\hline
2 -- 400 кГц& 2.5 В/м \\
\hline
\end{tabular}
\label{tab:U}
\end{table}

\begin{table}[ht]
\caption{Предельно допустимые значения плотности магнитного потока}
\begin{tabular}{|c|c|}
\hline
Диапазон частот& Допустимые значения \\
\hline
5 Гц -- 2 кГц & 250 нТл \\
\hline
2 -- 400 кГц& 5 нТл \\
\hline
\end{tabular}
\label{tab:U}
\end{table}

Величина поверхностного электрического потенциала не должна превышать 500 В.

Мониторы, используемые в настоящее время, удовлетворяют более жестким нормам MPR II, а значит и СанПиН.

\subsection{Визуальные параметры}

Неправильный выбор визуальных эргономических параметров приводит к ухудшению здоровья пользователей, быстрой утомляемости, раздражительности. В связи с этим, проектом предусмотрено, что конструкция вычислительной системы и ее эргономические параметры обеспечивают комфортное и надежное считывание иныформации. Требования к визуальным параметрам, их внешнему виду, дизайну, возможности настроцки представлены в СанПиН 2.2.2/2.4.1340--03. Визуальные эргономические параметры монитора и пределы из изменений приведены в таблице~\ref{ergonom}.

\begin{table}[ht]
\caption{Визуальные эргономические параметры ВДТ и пределы из изменений}
\begin{tabular}{|l|c|c|}
\hline
\multirow{2}{*}{Наименование параметров} & \multicolumn{2}{c|}{Пределы значений параметров}\\
\cline{2-3}
&не менее&не более\\
\hline
Яркость экрана (фона), $\frac{\text{кд}}{\text{м}^2}$ (измеренная в темноте) &35&120\\
\hline
Внешняя освещенность экрана, лк &100&250\\
\hline
Угловой размер экрана, угл.мин. &16&60\\
\hline
\end{tabular}
\label{tab:ergonom}
\end{table}

Для выполнения этих требований проектом предусмотренно использование современных мониторов, имеющих достаточно широкий набор регулируемых параметтров.  В частности, для удобного считывания информации реализована возможность настройки положения монитора по горизонтали и вертикали. Мониторы оснащены специальными устройствами и средствами настройки ширины, высоты, яркости, контраста и разрешения изображения. кроме того, в современных мониторах зерно изображения имеет размер в пределах 0.27 мм, что обеспечивает высокую четкость и непрерывность изображения. Наконец, на поверхность дисплея нанесено матовое покрытие, чтобы избавиться от солнечных бликов.

\section{Расчет искусственного освещения}

При расчете освещенности от светильников общего равномерного освещения наиболее часто применяют метод расчета по светтовому потоку. При расчете освещения по этому методу необходимое количество светильников для освещения оабочего места расчитывается по формуле:

\begin{equation}
\label{f:lightsCount}
N = \frac{E_{min}\cdot S\cdot K}{F_\text{Л} \cdot \text{З} \cdot z \cdot h}
\end{equation}

где $E_{min}$~--- нормируемая минимальная освещенность; $S$~--- площадь помещения, $\text{м}^2$; $F_\text{Л}$~--- ветовой поток лампы, лк; $K$~--- коэффициент запаса; $z$~--- коэффициент неравномерности освещения (для люминисцентных ламп~---1.1); $h$~--- коэффициент использования светового потока в долях единицы.

$E_{min}$ определяется на основании нормативного документа СНиП23--05--95. В соответствии с произведенным выбором в предыдущем разделе, для работы программиста $E_{min}=300$ лк (общее освещение).

Работы, производятся в помещении, требуют различения цветных объектов при невысоких требованиях к цветоразличению, поэтому в качестве источника освещения была выбрана лампа люминисцентная холодно-белая (ЛХБ), 1940 лк, 30 Вт. В помещениях общественных и жилых зданий с нормальными условиями среды: К=1.4.

Для люминисцентных ламп коэффициент неравномерности освещения Z=1.1.

Коэффициент использования h зависит от типа свтильника, от коэффициентов отражения потолка $\rho_\text{П}$, стен $\rho_\text{С}$, расчетной поверхности $\rho_\text{Р}$ и индекса помещения.

Высота подвеса над рабочей поверностю Нр=3 м. Размеры помещения А=3.5 м, В=3 м. Определим индекс по мещения по формуле:

\begin{equation}
\phi = \frac{A \cdot B}{H_P \cdot (A + B)} = \frac{3.5 \cdot 3}{s \cdot (3.5 + 3)} = 0.54
\end{equation}

Для светлого фона примем:$\rho_\text{П} = 70$ $\rho_\text{С} = 50$ $\rho_\text{Р} = 10$. h = 59\%.

Освещение проектируется при помощи светильников ОДОР с минимальной освещенностью $E_{min}=300$ лк, P=40 Вт. Число ламп в ОДОР равно 2. Необходимое число светильников для данной комнаты:

\begin{equation}
N = \frac{300 \cdot 9 \cdot 1.4}{1940 \cdot 0.59 \cdot 1.1 \cdot 2} = 2 \text{шт}
\end{equation}

Общее количество ламп $n = 2\times2=4$ шт. Длина светильника ОДОР=1.26 м. Поскольку длина помещения 3 м, то светильники помещаются в два ряда. схема размещения светильников показана на Рисунке~\ref{fig:light}.

\begin{figure}
\includegraphics[width=\textwidth]{inc/light.png}
\caption{План размещения светильников в машинном зале}
\label{fig:light}
\end{figure}

Суммарная мощность светильников: $30\cdot4=160$ Вт. Сумарный световой поток: $1940\cdot4=7760$ лм.

Режим труда и отдыха должен зависеть от характера работы: при вводе данных, раедактировании программ, чтении информации с экрана непрерывная продолжительность работы с монитором не должна превышать 4 часов. При 8 часовом рабочем дне, через кадый час рпботы необходимо проводить перерыв 5--10 минут, а каждые два часа перерыв в 15 мин.


\backmatter %% Здесь заканчивается нумерованная часть документа и начинаются ссылки и
            %% заключение

%\chapter{Заключение}

В результате проделанной рабыты был проведен обзор системы моделирования GPSS и решены следующие задачи:

\begin{itemize}
\item Обоснован выбор подмножества блоков GPSS, реализуемых в работе.
\item Разработан и реализован алгоритм формирования моделей.
\item Реализованы алгоритмы имитационного моделирования сформированных моделей.
\item Разработаны аналитическая и имитационная модель тестовой системы.
\item Реализована бибилиотека имитационного моделирования.
\item Реализована демонстрационная пограмма, позволяющая оценить возможности разработанной библиотеки и оценить точность выбранных меетодов.
\item Проведено сравнение результатов аналитического и имитационного моделирования тестовой системы, что подтвердило корректность реализации разработанной бибилиотеки.


\end{itemize}


\include{81-biblio}

\appendix   % Тут идут приложения
	

\end{document}

%%% Local Variables:
%%% mode: latex
%%% TeX-master: t
%%% End:
