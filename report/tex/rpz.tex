%% Преамбула TeX-файла

% 1. Стиль и язык
\documentclass[utf8x, 14pt]{G7-32} % Стиль (по умолчанию будет 14pt)

% Остальные стандартные настройки убраны в preamble-std.tex
\sloppy

% 1. Настройки стиля ГОСТ 7-32
% Для начала определяем, хотим мы или нет, чтобы рисунки и таблицы нумеровались в пределах раздела, или нам нужна сквозная нумерация.
% А не забыл ли автор букву 't' ?
\EqInChapter % формулы будут нумероваться в пределах раздела
\TableInChapter % таблицы будут нумероваться в пределах раздела
\PicInChapter % рисунки будут нумероваться в пределах раздела

% 2. Добавляем гипертекстовое оглавление в PDF
\usepackage[
bookmarks=true, colorlinks=true, unicode=true,
urlcolor=black,linkcolor=black, anchorcolor=black,
citecolor=black, menucolor=black, filecolor=black,
]{hyperref}

% 3. Изменение начертания шрифта --- после чего выглядит таймсоподобно.
% apt-get install scalable-cyrfonts-tex

\IfFileExists{cyrtimes.sty}
    {
        \usepackage{cyrtimespatched}
    }
    {
        % А если Times нету, то будет CM...
    }


% 4. Прочие полезные пакеты.
\usepackage{underscore} % Ура! Теперь можно писать подчёркивание.
                        % И нельзя использовать подчёркивание в файлах.
                        % Выбирай, но осторожно.

\usepackage{graphicx}   % Пакет для включения рисунков

\usepackage{tikz}

 % 5. Любимые команды
\newcommand{\Code}[1]{\textbf{#1}}

% 6. Поля
% С такими оно полями оно работает по-умолчанию:
% \RequirePackage[left=20mm,right=10mm,top=20mm,bottom=20mm,headsep=0pt]{geometry}
% Если вас тошнит от поля в 10мм --- увеличивайте до 20-ти, ну и про переплёт не забывайте:
\geometry{right=20mm}
\geometry{left=30mm}


% 7. Tikz
\usepackage{tikz}
\usetikzlibrary{arrows,positioning,shadows}

% 8 Листинги

\usepackage{listings}

% Значения по умолчанию
\lstset{
  basicstyle= \footnotesize,
  breakatwhitespace=true,% разрыв строк только на whitespacce
  breaklines=true,       % переносить длинные строки
%   captionpos=b,          % подписи снизу -- вроде не надо
  inputencoding=koi8-r,
  numbers=left,          % нумерация слева
  numberstyle=\footnotesize,
  showspaces=false,      % показывать пробелы подчеркиваниями -- идиотизм 70-х годов
  showstringspaces=false,
  showtabs=false,        % и табы тоже
  stepnumber=1,
  tabsize=4,              % кому нужны табы по 8 символов?
  frame=single
}

% Стиль для псевдокода: строчки обычно короткие, поэтому размер шрифта побольше
\lstdefinestyle{pseudocode}{
  basicstyle=\small,
  keywordstyle=\color{black}\bfseries\underbar,
  language=Pseudocode,
  numberstyle=\footnotesize,
  commentstyle=\footnotesize\it
}

% Стиль для обычного кода: маленький шрифт
\lstdefinestyle{realcode}{
  basicstyle=\scriptsize,
  numberstyle=\footnotesize
}

% Стиль для коротких кусков обычного кода: средний шрифт
\lstdefinestyle{simplecode}{
  basicstyle=\footnotesize,
  numberstyle=\footnotesize
}

% Стиль для BNF
\lstdefinestyle{grammar}{
  basicstyle=\footnotesize,
  numberstyle=\footnotesize,
  stringstyle=\bfseries\ttfamily,
  language=BNF
}

% Определим свой язык для написания псевдокодов на основе Python
\lstdefinelanguage[]{Pseudocode}[]{Python}{
  morekeywords={each,empty,wait,do},% ключевые слова добавлять сюда
  morecomment=[s]{\{}{\}},% комменты {а-ля Pascal} смотрятся нагляднее
  literate=% а сюда добавлять операторы, которые хотите отображать как мат. символы
    {->}{\ensuremath{$\rightarrow$}~}2%
    {<-}{\ensuremath{$\leftarrow$}~}2%
    {:=}{\ensuremath{$\leftarrow$}~}2%
    {<--}{\ensuremath{$\Longleftarrow$}~}2%
}[keywords,comments]

% Свой язык для задания грамматик в BNF
\lstdefinelanguage[]{BNF}[]{}{
  morekeywords={},
  morecomment=[s]{@}{@},
  morestring=[b]",%
  literate=%
    {->}{\ensuremath{$\rightarrow$}~}2%
    {*}{\ensuremath{$^*$}~}2%
    {+}{\ensuremath{$^+$}~}2%
    {|}{\ensuremath{$|$}~}2%
}[keywords,comments,strings]

% Подписи к листингам на русском языке.
\renewcommand*\thelstnumber{\oldstylenums{\the\value{lstnumber}}}
\renewcommand\lstlistingname{\cyr\CYRL\cyri\cyrs\cyrt\cyri\cyrn\cyrg}
\renewcommand\lstlistlistingname{\cyr\CYRL\cyri\cyrs\cyrt\cyri\cyrn\cyrg\cyri}

% Произвольная нумерация списков.
\usepackage{enumerate}

\usepackage{pdfpages}
\begin{document}

\frontmatter % выключает нумерацию ВСЕГО; здесь начинаются ненумерованные главы: реферат, введение, глоссарий, сокращения и прочее

% Команды \breakingbeforechapters и \nonbreakingbeforechapters
% управляют разрывом страницы перед главами.
% По-умолчанию страница разрывается.

% \nobreakingbeforechapters
% \breakingbeforechapters

\includepdf[pages=-]{inc/doc/titul.pdf}
\includepdf[pages=-]{inc/doc/task.pdf}
\includepdf[pages=-]{inc/doc/plan.pdf}



\tableofcontents


<<<<<<< HEAD
=======

>>>>>>> 3cd846a8b8272c660d466065be5a359f58138830
\Introduction

Зародившаяся в начале прошлого века с целью упорядочить работу телефонных станций, теория массового обслуживания нашла применения в моделировании самых разнообразных систем, таких как системы связи, обработки информации, снабжения, производства и др.

Несмотря на имеющиеся достижения в области математического исследования характеристик систем массового обслуживания, наиболее универсальным подходом по прежнему остается имитационное моделирование.

Язык имитационного моделирования GPSS создан специально для моделирования систем массового обслуживания и на данный момен является доминирующим в этой области. Однако, существующие версии систем имитационного моделирования на основе языка GPSS либо слишком дороги, либо ограничены в возможностях и не позволяют провести все необходимые исследования.\cite{KST} Помимо этого, на данный момент затруднено интегрирование моделей, разработанных при помощи GPSS в другие программные средства (напимер, в целях оптимизации параметров исследуемой системы).

Целью данной работы является создание системы имитационного моделирования, основанной на принципах и синтаксисе GPSS, однако позволяющей разрабатывать модели как часть более крупной программы.

В качестве языка разработки был выбран Haskell. Haskell является динамично развивающимся функциональным языком проограммирования, который получает все больше сторонников во всем мире, в том числе и в России. \cite{HaskellRef}. Для Haskell характерны строгая статическая типизация, модульность, строгое разделение функций на чистые и не чистые, ленивые вычисления, функции высших порядков и др.\cite{Haskell} Помимо этого использование языка Haskell позволит производить описание систем при помощи синтаксиса схожего с синтаксисом GPSS, при этом разработанные модели будут являться объектами первого класса, что позволит, например, передать модель как параметр в функцию оптимизации.

Для достижения поставленной цели необходимо решить следующие задачи:
\begin{itemize}
\item изучить принципы функционирования и синтаксис описания моделей в GPSS;
\item разработать синтаксис описания моделей схожий с синтаксисом GPSS, но при этом позволяющий составлять модели в виде функций языка Haskell;
\item выбрать подмножество блоков GPSS, которые следует реализовать в системе;
\item реализовать алгоритмы описания моделей и имитационного моделирования;
\item разработать и реализовать транслятор моделей GPSS в формат разработанной системы моделиования;
\item провести тестирование разработанного программного обеспечения;
\item провести моделирование некоторой эталонной системы массового обслуживания в разработанной системе, GPSS и аналитически и убедиться в совпадении полученных результатов.
\end{itemize}



\mainmatter % это включает нумерацию глав и секций в документе ниже


\chapter{Аналитический раздел}
\label{cha:analysis}

В данном разделе проводится обзор принципов функционирования и синтаксиса системы GPSS, а также производится выбор подмножества возможностей GPSS, которые следует реализовать в разрабатываемой системе.

\section{Краткий обзор GPSS}

GPSS стал одним из первых языков моделирования, облегчающих процесс написания имитационных программ. Он был создан в виде конечного продукта Джефри Гордоном в фирме IBM в 1962~г.\cite{ImitGPSS} В свое время он входил в десятку лучших языков программирования и по сей день широко используется для решения практических задач. Наиболее современной версией GPSS для персональных компьютеров на данный момент является пакет GPSS World, разработанный компанией Minuteman Software.

Описание системы на GPSS представляет собой последовательность блоков,
каждый из которых соответствует некоторому оператору (подпрограмме). Каждый блок
имеет определенное количество параметров (полей).


Основой имитационных алгоритмов GPSS является дискретно-событийный подход~--- моделирование системы в дискретные моменты времени, когда происходят события, отражающие последовательность изменения состояний системы во времени.\cite{ImitGPSS}

\section{Объекты языка GPSS}

Основными Объектами языка GPSS являются транзакты и блоки, которые отображают соответственно динамические и статические объекты моделируемой системы.

Транзакты~--- динамические элементы GPSS-модели. В реальной системе транзактам могут соответствовать такие элементы как заявка, покупатель автомобиль и др. Состояние транзакта в процессе моделирования характеризуется следующими атрибутами:

\begin{enumerate}
\item параметры~--- набор значений связанных с транзактом. Каждый транзакт может иметь произвольное число параметров. Каждый параметр имеет уникальный номер, по которому на него можно сослаться;
\item приоритет~--- определяет порядок продвижения транзактов при конкурировании за общий ресурс;
\item текущий блок~--- номер блока, в котором транзакт находится в данный момент;
\item следующий блок~--- номер блока, в который транзакт попытается войти;
\item время появления транзакта~---  момент времени в который транзакт был создан;
\item состояние~--- состояние, показывающее в каких списках транзакт находится в данный момент. Транзакт может находиться в одном из следующих состояний:
    \begin{enumerate}
    \item активен~--- транзакт находится в списке текущих событий и имеет наивысший приоритет;
    \item приостановлен~--- транзакт находится в списке будущих событий либо в списке текущих событий, но с меньшим приоритетом;
    \item пассивен~--- транзакт находится в списке прерываний, списке синхронизации, списке блокировок или списке пользователя;
    \item завершен~--- транзакт уничтожен и больше не участвует в модели.
    \end{enumerate}
    Диаграмма состояний транзакта показана на Рисунке~\ref{fig:transactionState}.
\end{enumerate}


\begin{figure}[ht]
  \centering
  \includegraphics[width=\textwidth]{inc/dia/transactionState}
  \caption{Состояния транзакта}
  \label{fig:transactionState}
\end{figure}


Блоки~--- статические элементы GPSS-модели. Модель в GPSS может быть представлена как диаграмма блоков, т.е. ориентированный граф, узлами которого являются блоки, а дугам~--- направления движения транзактов. с каждым блоком связано некоторое действие, изменяющее состояние прочих элементов модели. Транзакты проходят блоки один за другим, до тех пор пока не достигнут блока TERMINATE. В ряде случаев транзакт может быть остановлен в одном из блоков до наступления некоторого события.


Помимо транзактов и блоков в GPSS используются следующие объекты: устройства, многоканальные устройства (хранилища, памяти), ключи, очереди, списки пользователя и др.


\section{Управления процессом моделирования в GPSS}

В системе GPSS интерпретатор поддерживает сложные структуры организации списков (см. Рисунок~\ref{fig:GPSSChains}).\cite{ImitGPSS} Два основных из них~--- список текущих событий (СТС) и список будущих событий (СБС).

\begin{figure}[ht]
  \centering
  \includegraphics[width=\textwidth]{inc/dia/gpss}
  \caption{Списки GPSS}
  \label{fig:GPSSChains}
\end{figure}

В СТС входят все события запланированные  на текущий  момент модельного времени. Интерпретатор в первую очередь просматривает этот список и перемещает по модели те транзакты, для которых выполнены все условия. Если таких транзактов в списке не оказалось интерпретатор обращается к СБС. Он переносит все события, запланированные на ближайший момент времени и вновь возвращается к просмотру СТС. Перенос также осуществляется в случае совпадения текущего момента времени с моментом наступления ближайшего события из СБС.

В целях эффективной организации просмотра транзактов, движение которых заблокировано (например, из-за занятости некоторого ресурса), используются следующие вспомогательные списки:

\begin{itemize}
\item списки блокировок~--- списки транзактов, которые ожидают освобождения некоторого ресурса;
\item список прерываний~--- содержит транзакты, прерванные во время обслуживания. Используется для организации обслуживания одноканальных устройств с абсолютным приоритетом;
\item списки синхронизации~--- содержат транзакты одного семейства (созданные блоком SPLIT), которые ожидают синхронизации в блоках (MATCH, ASSEMBLE или GATHER);
\item списки пользователя~--- содержат транзакты, выведенные пользователем из СТС с помощью блока LINK. Транзакты могут быть возвращены в СТС помощью блока UNLINK.
\end{itemize}




\section{Выбор подмножества реализуемых блоков}

В современной версии языка GPSS (входящей в пакет GPSS World) поддерживается 53 различных блока.\cite{GPSSRef} В рамках данной работы не представляется возможным реализовать  аналоги каждого из них. Поэтому следует выделить некоторое подмножество блоков, которое с одной не будет слишком обширным, а с другой~--- позволит решать практические или по крайней мере учебные задачи.

В качестве примера рассмотрим задачу из курса Модели оценки качества аппаратно программных комплексов:

\begin{quote}
В вычислительной системе, содержащей N процессоров и M каналов обмена данными, постоянно находятся K задач. Разработать модель, оценивающую производительность системы с учетом отказов и восстановлений процессоров и каналов. Имеется не более L ремонтных бригад, которые ремонтируют отказывающие устройства с бесприоритетной  дисциплиной. Интенсивность отказов, восстановлений, средние времена обработки сообщения и среднее время обдумывания также известны.
\end{quote}

Схема модели данной системы показана на Рисунке~\ref{fig:mainModel}

\begin{figure}[ht]
\centering
\includegraphics[width=\textwidth]{inc/dia/main}
\caption{Схема моделируемой системы}
\label{fig:mainModel}
\end{figure}

Как и подавляющее большинство других задач, данная задача, безусловно, не может быть решена без использования блоков GENERATE, TERMINATE и ADVANCE. Так как моделируемая система является замкнутой, при описании модели не обойтись без блока TRANSFER.

К сожалению, не представляется возможным реализовать процессоры и каналы как многоканальные устройства, т.к. многоканальные устройства в GPSS не поддерживают абсолютные приоритеты и не позволяют смоделировать выход из строя отдельных каналов устройства. Однако, требуемую систему можно описать при помощи множества одноканальных устройств и блока TRANSFER в режиме ALL. Таким образом, также понадобятся блоки SEIZE и RELEASE. Для моделирования отказов устройств можно воспользоваться блоками FAVAIL и FUNAVAIL либо блоками PREEMPT и RETURN.

Наконец, доступные ремонтные бригады можно смоделировать с помощью многоканального устройства. Соответственно, понадобятся блоки ENTER и LEAVE.

Приблизительная модель системы показана в Листинге~\ref{lst:sample01}

\lstinputlisting[caption=Приблизительная модель системы,label=lst:sample01]{inc/src/analysModel.gpss}

Таким образом, разрабатываемая система имитационного моделирования должна поддерживать аналоги по крайней мере следующих блоков: ADVANCE, ENTER, GENERATE, LEAVE, PREEMPT, RELEASE, RETURN, SEIZE, TERMINATE и TRANSFER.

\section{Описание выбранных блоков}

Ниже представлено описание выбранных блоков в соответствии со справочным руководством GPSS World.\cite{GPSSRef}

\subsection*{ADVANCE A,B}

Блок ADVANCE осуществляет задержку продвижения транзактов на заданный промежуток времени.

A~--- Среднее время задержки. Не обязательный параметр. Значение по умолчанию~---~0.

B~--- Максимально допустимое отклонение времени задержки либо функция-модификатор.

\subsection*{ENTER A,B}

При входе в блок ENTER транзакт либо занимает заданное количество каналов указанного многоканального устройства либо блокируется до его освобождения.

A~--- Имя или номер многоканального устройства. Обязательный параметр.

B~--- Число требуемых каналов. Не обязательный параметр. Значение по умолчанию~--- 1.

\subsection*{GENERATE A,B,C,D,E}

Блок GENERATE предназначен для создания новых транзактов.

A~--- Среднее время между генерацией последовательных заявок. Не обязательный параметр.

B~--- Максимальное допустимое отклонение времени генерации либо функция-модификатор. Не обязательный параметр.

С~--- Задержка до начала генерации первого транзакта. Не обязательный параметр.

D~--- Ограничение на максимальное допустимое число созданных транзактов. Не обязательный параметр. Пол умолчанию ограничение отсутствует.

E~--- Уровень приоритета создаваемых заявок. Не обязательный параметр. Значение по умолчанию~--- 0.

\subsection*{LEAVE A,B}

При входе в блок LEAVE транзакт освобождает заданное число каналов указанного многоканального устройства.

A~--- Имя или номер многоканального устройства. Обязательный параметр.

B~--- Число требуемых каналов. Не обязательный параметр. Значение по умолчанию~--- 1.

\subsection*{PREEMPT A,B,C,D,E}

Блок PREEMPT подобен блоку SEIZE и вошедший в него транзакт также пытается занять указанное одноканальное устройство. Однако, данный блок позволяет транзакту занять устройство, даже если в данный момент оно уже занято другим транзактом, при соблюдении ряда условий, определяемых параметрами блока.

A~--- Имя или номер одноканального устройства. Обязательный параметр.

B~--- задает режим работы блока. PR~--- режим приоритетов. По умолчанию~--- режим прерываний. В режиме прерываний транзакт может вытеснить из устройства любой другой транзакт, если тот в свою очередь не захватил устройство через блок PREEMPT. В режиме приоритетов транзакт может вытеснить любой транзакт с меньшим приоритетом.

C~--- задает номер блока, куда будет направлен транзакт вытесненный и устройства в результате действия блока PREEMPT. 

D~--- задает номер параметра вытесненного транзакта, в  котором будет сохранено время, которое осталось транзакту до окончания обработки а устройстве.

E~--- задает режим удаления вытесненного транзакта. RE~--- вытесненное сообщение удаляется из устройства и более не претендует на владение им. Требует обязательного указания параметра C. Значение по умолчанию~--- вытесненный транзакт будет вновь пытаться занять устройство. 

\subsection*{RELEASE A}

Блок RELEASE освобождает одноканальное устройство.

A~--- Имя или номер одноканального устройства. Обязательный параметр.

\subsection*{RETURN A}

Блок RELEASE освобождает одноканальное устройство.

A~--- Имя или номер одноканального устройства. Обязательный параметр.

\subsection*{SEIZE A}

При входе в блок SEIZE транзакт занимает указанное одноканальное устройство либо блокируется до его освобождения.

A~--- Имя или номер одноканального устройства. Обязательный параметр.

\subsection*{TERMINATE A}

Блок TERMINATE завершает поступивший в него транзакт. И опционально уменьшает счетчик завершенных транзактов. Когда счетчик достигает нуля имитация останавливается.

A~--- Значение, на которое следует уменьшить счетчик завершенных транзактов. Не обязательный параметр. Значение по умолчанию~--- 0.

\subsection*{TRANSFER A,B,C,D}

Блок TRANSFER является основным средством позволяющим изменить маршрут транзакта и перенаправить его к произвольному блоку модели. Параметр A определяет режим работы блока. Смысл остальных параметров меняется в зависимости от выбранного режима.

\paragraph{Безусловный режим}. Если параметр А пропущен, то блок TRANSFER работает в безусловном режиме. Входящий в блок TRANSFER транзакт переходит к блоку, указанному в поле В.

\paragraph{Статистический режим}. Параметр А является числом от 0 до 1, показывающим какая доля транзактов перейдет к блоку, указанному в параметре С. Остальные транзакты переходят к блоку,ь указанному в параметре B.

\paragraph{Режим BOTH}. Если параметр А равен BOTH, то блок TRANSFER работает в одноименном режиме.В этом режиме каждый входящий транзакт сначала пытается перейти к блоку, указанному в поле В. Если это сделать не удается, транзакт пытается перейти к блоку, указанному в поле С. Если транзакт не сможет перейти ни к тому, ни к другому блоку, он остается в блоке TRANSFER и будет повторять в том порядке попытки перехода при каждом просмотре списка текущих событий, до тех пор, пока не сможет выйти из блока TRANSFER.

\paragraph{Режим ALL}. Если параметр А равен ALL, то блок TRANSFER работает в одноименном режиме.В этом режиме каждый входящий транзакт1 прежде всего пытается перейти к блоку, указанному в поле В. Если транзакт не может войти в этот блок, то последовательно проверяются все блоки в определенном ряду в поисках первого, способного принять это сообщение, включая последний блок, указанный операндом С. Номер каждого проверяемого блока вычисляется как сумма номера предыдущего блока и шага, заданного операндом D. По умолчанию значение операнда D принимается равным 1.

\paragraph{Режим PICK}. Если параметр А равен PICK, то блок TRANSFER работает в одноименном режиме. Этот режим подобен режиму ALL, за тем исключением, что блок назначения выбирается случайным образом с одинаковой вероятностью.

\section{Выводы}

Был проведен обзор устройства системы GPSS и осуществлен выбор подмножества блоков, необходимых для моделирования не сложных систем массового обслуживания. Представлено описание назначения и параметров каждого из выбранных блоков.



%\chapter{Конструкторский раздел}

В данном разделе проводится выбор синтаксиса описания моделей в разрабатываемой системе, описываются алгоритмы и структуры данных, используемые при формировании моделей и непосредственно при моделировании, а также проводится построение аналитической и имитационной модели учебной системы описанной в предыдущем разделе.



\section{Требования к синтаксису}

Синтаксис разрабатываемой системы должен быть, на сколько это возможно, схож с синтаксисом системы GPSS. 

Программа на языке GPSS представляет из себя последовательность операторов, каждый из которых описывает тот или иной элемент модели (функцию, блок, устройство и др.). Этот подход естественен для императивных языков программирования, в которых программа является последовательностью команд, меняющих состояние программы. Однако Haskell относится к категории функциональных языков, программы на которых описываются как функции, значение которых вычисляется. При этом нет фиксированной, заданной программистом, последовательности операций, которые должны быть выполнены для достижения результата. 

Тем не менее, в языке Haskell предусмотрен механизм, позволяющий описать конкретную последовательность вычислений~--- монады. В сочетании с так называемой do-нотацией, этот механизм позволит проводить описание моделей на Haskell, используя синтаксис схожий с GPSS.

\section{Монады}

Понятие монады в языке Haskell основано на теории категорий. В рамках данной теории монада может быть определена (не вполне строго) как моноид в категории эндофункторов. Однако для практического использования этого понятия в рамках языка Haskell можно обойтись менее формальным определением.

В соответствии с \cite{Haskell} монада~--- это контейнерный тип данных (то есть такой, который содержит в себе значения других типов), представляющий собой экземпляр класса Monad определенного в модуле Prelude. 

Под классом в Haskell, понимается не тип данных, как в объектно-ориентированных языках, а набор методов (функций), которые применимы для работы с теми или иными типами данных, для которых объявлены экземпляры заданных классов. Наиболее близким аналогом классам в Haskell являются интерфейсы в таких языках как Java или C\#. Более точно их следует называть классами типов, но т.к. в данной работе используется исключительно функциональная парадигма, в дальнейшем для краткости они будут называться просто классами.

Значения монад можно воспринимать, как значения м некоторым дополнительным контекстом. В случае монады Maybe значения обладают дополнительным контекстом того, что вычисления могли закончиться неуспешно. Монада IO добавляет контекст, указывающий что получение значений связано с действиями ввода/вывода и потому не является детерминированным и может иметь побочные эффекты. В случае списков (которые также являются монадой) контекстом является то, что значение может являться множественным или отсутствовать.

Класс Monad определен в модуле Prelude следующим образом:

\lstinputlisting[caption=Класс Monad,label=lst:monad]{inc/src/Monad.hs}

Функция \Code{return}\footnote{Следует отметить, что название return никак нельзя назвать удачным, так как оно неизбежно вызывает ассоциации с одноименным оператором из многих императивных языков программирования, на которые она не похожа ничем кроме названия. Данная функция не завершает выполнение функции, а лишь оборачивает переданное значение в монаду.} преобразует переданное ей значение типа a  в монадическое значение типа m a. Другими словами она помещает значение в некоторый контекст по умолчанию, в зависимости от выбранной монады. Для списка это будет список из одного элемента, для монады IO~--- действие ввода вывода, всегда возвращающее заданное значение и не имеющее побочных эффектов и т.д.

Функция \Code{$>>=$} определяет операцию связывания. Она принимает монадическое значение и передает его функции, которая принимает обычное значение и возвращает монадическое. При этом сохраняется накопленный контекст и к нему добавляется новый, полученный в результате выполнения функции.

Функция \Code{$>>$} также предназначена для связывания и используется в тех случаях, когда переданное монадическое значение не представляет интереса, а значение имеет только переданный с ним контекст вычислений. Для этой функции в классе определена реализация по умолчанию, по этому в большинстве случаев при определении экземпляра класса Monad в явном виде ее не реализуют.

Функция \Code{fail} никогда не вызывается программистом явным образом и предназначена для обработки неуспешного окончания вычислений при сопоставлении с образцом в do-нотации, что позволяет избежать аварийного завершения программы и вернуть неудачу в контексте текущей монады.

\section{Нотация do}

Так как монады находят крайне широкое применение в программах на языке Haskell (в первую очередь, без использования монады IO невозможно осуществить ввод/вывод), в синтаксис языка было добавлено специальное ключевое слово \Code{do}, призванное упростить написание монадических функций, сделать их более читаемыми и избавить от излишнего <<синтаксического мусора>>.

Если в коде программы встречается конструкция с ключевым словом \Code{do}, то транслятор выполняет следующие преобразования\footnote{В приведенных преобразованиях используются управляющие символы \Code{;}, \Code{\{} и \Code{\}}, хотя в реальных программах на языке Haskell их можно встретить довольно редко. Это связано с тем, что в Haskell используется так называемый <<двумерный синтаксис>>: при правильной расстановке отступов, транслятор самостоятельно расставляет точки с запятой и фигурные скобки и в большинстве случаев нет смысла загромождать ими исходный код. Тем не менее в случае необходимости их можно расставить и явным образом.}:

\begin{verbatim}
1. do {e} → e
2. do {e; es} → e >> do {es}
3. do {let decls; es} → let decls in do {es}
4. do {p <- e; es} → let ok p = do {es}
                          ok _ = fail "..."
                      in e >>= ok
\end{verbatim}

При помощи нотации \Code{do} приведенный ниже фрагмент кода

\begin{verbatim}
foo :: Maybe String
foo = Just 3 >>= (\x -> Just "!" >>= (\y -> Just (show x ++ y)))}
\end{verbatim}

может быть записан в следующей более читаемой форме:

\begin{verbatim}
foo :: Maybe String
foo = do x <- Just 3 
         y <- Just "!" 
         return (show x ++ y)
\end{verbatim}

\section{Монада State}

Часто в процессе вычислений возникает необходимость хранить и изменять некоторое состояние, в зависимости от которого результат вычислений может меняться. Haskell является чистым функциональным языком программирования функции должны быть детерминированы и не иметь побочных эффектов, поэтому текущее состояние обычно передается в функции как еще один параметр, а возвращает функция пару из собственно результата и обновленного состояния.

Для того, чтобы упростить написание функций оперирующих некоторым состоянием в Haskell была введена монада \Code{State}. Она определена в модуле \Code{Control.Monad.State} следующим образом:

\begin{verbatim}

newtype State s a = State {runState :: s -> (a, s)}

instance Monad (State s) where
    return x = State $ \s -> (x,s)
    (State h) >>= f = State $ \s -> let (a, newState) = h s
                                        (State g) = f a
                                    in g newState

\end{verbatim}

Функция \Code{return} создает вычисление с состоянием, которое всегда возвращает один и тот же результат и оставляет переданное в него состояние без изменений. Функция \Code{$>>=$} <<склеивает>> два вычисления с состоянием так, что конечное  состояние первого становится начальным для второго, а результат и конечное состояние второго вычисления становятся также результатом и конечным состоянием итогового, составного вычисления.

Помимо этого для работы с монадой \Code{State} используются две вспомогательные функции \Code{put} и \Code{get}. Функция \Code{put} является вычислением, которое устанавливает состояние в заданное значение не зависимо от его предыдущего значения и не возвращает никакого результата (точнее возвращает кортеж нулевой длины \Code{()}). Функция \Code{get} возвращает текущее состояние и оставляет его без изменений. 

\section{Описание модели как вычисление с состоянием}

Описание модели на языке GPSS представляет из себя последовательность блоков. В Haskell такое описание удобно представить как последовательность функций, каждая из которых добавляет к уже сформированной модели очередной блок. 

Такой процесс удобно представить как вычисление с состоянием. Каждая функция, формирующая блок, помимо параметров самого блока должна принимать текущее состояние~--- список уже сформированных к данному моменту боков в порядке их формирования. В качестве результата функция возвращает новое состояние~--- модель к которой добавлен только что сформированный блок (см. Рисунок \ref{fig:StateIDEF}).

\begin{figure}[ht]
  \centering
  \includegraphics[width=\textwidth]{inc/dia/StateIDEF}
  \caption{Процесс формирования модели}
  \label{fig:StateIDEF}
\end{figure}

Для реализации такого механизма целесообразно воспользоваться монадой \Code{State}, что позволит скрыть явную передачу состояния от одной функции к другой. А использование нотации \Code{do} сделает описание модели почти идентичным синтаксису GPSS:

\begin{verbatim}

model = 
    do generate (10,2)
       advance 3
       terminate 1

\end{verbatim}


\section{Функции формирования блоков}

В языке GPSS имена всех блоков пишутся с заглавной буквы. Параметры отделяются от имени блока пробелом и разделяются запятыми. Синтаксис Haskell не позволяет в точности повторить эти соглашения. Имена функций в Haskell обязаны начинаться со строчной буквы. Параметры функций обычно разделяются пробелами и не берутся в скобки (каррированные функции) либо заключаются в скобки и разделяются запятыми (не каррированные функции)\footnote{Строго говоря все функции в Haskell принимают ровно один параметр. Функции от N параметров на самом деле принимают один параметр и возвращают функцию от N-1 параметра (каррированные функции) либо принимают параметр-кортеж (некаррированные).}. Оба варианта описания параметров одинаково близки к синтаксису GPSS и можно выбрать любой из них, однако для второго варианта существенно легче реализовать перегрузку функций.

\section{Состояние транзакта}

Основным объектом в процессе моделирования является транзакт. Моделирование представляет собой передвижение транзактов от блока к блоку, в процессе которого могут изменяться состояния тех или иных объектов системы (обслуживающих аппаратов, очередей, хранилищ).

Параметры, определяющие состояние транзакта приведены в таблице~\ref{tab:transactionState}

\begin{table}
\caption{Параметры состояния транзакта}
\label{tab:transactionState}
\begin{tabular}{|l|l|p{0.6\textwidth}|}
\hline
Имя параметра & Тип &Описание \\
\hline
currentBlock & Int & Номер блока модели, в котором в данный момент находится транзакт.\\
\hline
nextBlock & Int & Номер блока в который транзакт попытается перейти.\\
\hline
priority & Int & Приоритет транзакта.\\
\hline
params & IntMap Double & Массив параметров транзакта.\\
\hline
ownership & String & Имя устройства, на котором в данный момент обрабатывается транзакт.\\
\hline
\end{tabular}
\end{table}

Также как часть состояния транзакта моет рассматриваться информация о том, в каком из глобальных или локальных списков событий он находится в данный момент. Так как в каждый момент транзакт должен находиться только в одном из списков, хранить ту информацию в отдельном поле не целесообразно.

\section{Состояния объектов системы}

В процессе перемещения по блокам транзакты изменяют состояния других объектов системы. Эти состояния во-первых, в свою очередь, оказывает влияние на движение транзактов, а во-вторых предназначено для сбора статических данных в процессе моделирования.

Параметры состояний обслуживающих аппаратов, хранилищ и очередей приведены в таблицах~\ref{tab:facState},~\ref{tab:storState} и \ref{tab:queueState} соответственно.

\begin{table}
\caption{Параметры состояния обслуживающего аппарата}
\label{tab:facState}
\begin{tabular}{|l|l|p{0.6\textwidth}|}
\hline
Имя параметра & Тип & Описание \\
\hline
idAvailable & Bool & Флаг, показывающий доступно или занято в данный момент обслуживающее устройство\\
\hline
toInterupted & Bool & Флаг, показывающий, захватил ли, обрабатывающийся в данный момент транзакт, устройство обычным образом или путем вытеснения обрабатывавшегося до этого транзакта.\\
\hline
captureCount & Int & Счетчик, показывающий сколько раз было захвачено данной устройство.\\
\hline
captureTime & Double & Суммарное время, в течение которого устройство было занято.\\
\hline
lastCaptureTime & Double & Момент времени, когда устройство было захвачено в последний раз.\\
\hline
utilization & Double & Процент времени, в течении которого устройство было занято.\\
\hline
ownerPriority & Int & Приоритет транзакта, обслуживающегося на устройстве в данный момент. \\
\hline
dc & [Transaction] & Список заявок, ожидающих освобождения устройства. \\
\hline
ic & \parbox{25mm}{[(Maybe Double,\\Transaction)] }  & Список заявок, вытесненных с устройства и ожидающих его освобождения для продолжения обслуживания.\\
\hline
pc & [Transaction] & Список заявок, не сумевших вытеснить, обрабатываемый в данный момент транзакт, и ожидающих освобождения устройства.\\
\hline
\end{tabular}
\end{table}


\begin{table}
\caption{Параметры состояния хранилища}
\label{tab:storState}
\begin{tabular}{|l|l|p{0.65\textwidth}|}
\hline
Имя параметра & Тип & Описание \\
\hline
capacity & Int & Емкость хранилища.\\
\hline
unused & Int & Число доступных единиц ресурса в хранилище.\\
\hline
avgInUse & Double & Среднее число занятых единиц ресурса.\\
\hline
useCount & Int & Число захватов ресурса за время моделирования.\\
\hline
lastMod & Double & Момент последнего захвата или освобождения ресурса.\\
\hline
utilization & Double &Средний процент захваченных единиц ресурса.\\
\hline
maxInUse & Int & Максимальное число одновременно захваченных единиц ресурса. \\
\hline
dc & [Transaction] & Список заявок, ожидающих освобождения достаточного числа ресурсов хранилища. \\
\hline
\end{tabular}
\end{table}


\begin{table}
\caption{Параметры состояния очереди}
\label{tab:queueState}
\begin{tabular}{|l|l|p{0.65\textwidth}|}
\hline
Имя параметра & Тип & Описание \\
\hline
currentContent & Int & Число заявок в очереди в данный момент модельного времени.\\
\hline
maximumContent & Int & Максимальное число заявок в очереди за все время моделирования.\\
\hline
averageContent & Double & Среднее число заявок в очереди.\\
\hline
lastChangeTime & Double & Момент времени, когда в последний раз заявка встала в очередь или покинула ее.\\
\hline
\end{tabular}
\end{table}


\section{Состояние системы в целом}

В процессе имитационного моделирования система последовательно переходит из одного состояния в другое, до тех пор, пока не будет достигнуто некоторое условие остановки моделирования. В данном случае моделирование происходит до тех пор, пока в блоках TERMINATE не будет завершено заданное число транзактов.

Состояние моделируемой системы может быть описано параметрами, указанными в таблице~\ref{tab:simState}. Моделирование продолжается до тех пор, пока значение параметра \Code{toTerminate} не достигнет нуля.

Отношения перечисленных сущностей показаны на рисунке~\ref{fig:umlSim}.

\begin{figure}[ht]
  \centering
  \includegraphics[width=\textwidth]{inc/dia/er}
  \caption{Отношения между сущностями системы}
  \label{fig:umlSim}
\end{figure}


\begin{table}[ht!]
\caption{Параметры состояния системы}
\label{tab:simState}
\begin{tabular}{|l|l|p{0.5\textwidth}|}
\hline
Имя параметра & Тип & Описание \\
\hline
currentTime & Double & Текущий момент модельного времени.\\
\hline
toTerminate & Int & Число транзактов, которое необходимо завершить для окончания моделирования.\\
\hline
blocks & Array Int SBlock & Список блоков, составляющих модель.\\
\hline
facilities & Map String Facility & Список состояний обслуживающих аппаратов.\\
\hline
storages & Map String Storage & Список состояний хранилищ.\\
\hline
queues & Map String Queue & Список состояний очередей.\\
\hline
cec & [Transaction] & Список будущих событий~--- список транзактов, продвижение которых требует наступления некоторого момента модельного времени. Упорядочен по возрастанию ожидаемого момента времени. \\
\hline
fec & [(Double,Transaction)] & Список текущих событий~--- список транзактов, продвижение которых возможно в данный момент модельного времени. Упорядочен по убыванию приоритета транзактов. \\
\hline
\end{tabular}
\end{table}


\section{Алгоритм имитационного моделирования}

Процесс моделирования запускается при вызове функции, одним параметром которой является сформированная модель, а вторым~--- количество транзактов, которое необходимо завершить для окончания моделирования.  Схема алгоритма показана на рисунке~\ref{fig:simFlowchart}.

\begin{figure}[ht!]
  \centering
  \includegraphics[height=0.7\textheight]{inc/dia/simFlowchart}
  \caption{Алгоритм имитационного моделирования}
  \label{fig:simFlowchart}
\end{figure}

На первом шаге алгоритма происходит активация всех блоков GENERATE. Для каждого из них вычисляется время создания ближайшего транзакта и эти транзакты помещаются в список будущих событий.

На втором шаге из списка будущих событий извлекаются события, наступающие в ближайший момент модельного времени. Модельное время передвигается на момент наступления тих событий, а сами события помещаются в  список текущих событий. 

На третьем шаге, до тех пор пока список текущих событий не опустеет, происходит продвижение транзакта из того списка с наибольшим приоритетом. Продвижение каждого транзакта происходит до тех пор, пока транзакт тем или иным образом не покинет текущий список (например, войдет в блок ADVANCE и будет помещен в список будущих событий или попытается войти в блок занятого устройства и попадет в список транзактов, ожидающих освобождения того устройства).

Шаги два и три повторяются до тех пор, пока в процессе моделирования не будет завершено заданное число транзактов.




\section{Обработка захода транзакта в блок}

Каждый раз, когда заявка пытается зайти в очередной блок, вызывается функция обработчик, которая определяет, может ли транзакт это сделать и какие дополнительные действия при этом должны быть выполнены. Исключением является блок GENERATE, для которого обработчик (определяющий время создания нового транзакта) вызывается при выходе из него транзакта. 

Обработчики индивидуальны для каждого типа блоков. В качестве примера на рисунках~\ref{fig:enterFlowchart} и \ref{fig:leaveFlowchart} показаны алгоритмы обработчиков блоков ENTER и LEAVE соответственно.

\begin{figure}[ht!]
  \centering
  \includegraphics[width=\textwidth]{inc/dia/enterFlowchart}
  \caption{Алгоритм обработки блока ENTER}
  \label{fig:enterFlowchart}
\end{figure}

\begin{figure}[ht!]
  \centering
  \includegraphics[width=\textwidth]{inc/dia/leaveFlowchart}
  \caption{Алгоритм обработки блока LEAVE}
  \label{fig:leaveFlowchart}
\end{figure}

При входе транзакта в блок ENTER проверяется количество свободных единиц ресурса в соответствующем хранилище. Если ресурса достаточно, то транзакт успешно входит в блок, количество доступных ресурсов уменьшается и обновляется статистика использования хранилища. Затем транзакт продолжает движение по блокам. Если же транзакту требуется больше ресурсов, чем в данный момент есть в хранилище, транзакт попадает в список транзактов, ожидающих освобождения ресурсов.

При входе транзакта в блок LEAVE освобождается указанное количество ресурсов соответствующего хранилища и транзакт продолжает движение по блокам. Если есть транзакты, ожидающие освобождения ресурса, среди них выбирается транзакт с наибольшим приоритетом и делается попытка выделить ему необходимое количество ресурса. Если это удается, то транзакт помещается в список текущих событий, в противном случае он возвращается в список ожидания.

\section{Структура библиотеки}

На рисунке~\ref{fig:libStruct} показана общая структура спроектированной библиотеки.

\begin{figure}[ht!]
  \centering
  \includegraphics[width=\textwidth]{inc/dia/libStructHuge}
  \caption{Структура разработанной библиотеки}
  \label{fig:libStruct}
\end{figure}

На диаграмме можно выделить:


\begin{itemize}
\item {Модуль содержащий непосредственно алгоритм имитационного моделирования.}
\item {Группу модулей с обработчиками захода транзактов в те или иные блоки.}
\item {Модуль формирующий результаты моделирования на основе заключительного состояния системы.}
\item {Группу модулей предназначенных для формирования моделируемой системы и содержащие функции добавляющие в систему те или иные блоки.}

\end{itemize}

\section{Демонстрационная программа}

Для демонстрации возможностей спроектированной библиотеки, а также с целью удостовериться  в адекватности выбранных алгоритмов моделирования и верности их реализации, целесообразно разработать демонстрационную программу. 

Проектируемая программа должна проводить решение приведенной в предыдущем разделе задачи аналитически и при помощи разработанной библиотеки при различных входных параметрах и выводить результаты в удобной для сравнения форме. Целесообразно также предусмотреть возможность автоматического варьирования выбранного параметра модели и построения графика зависимости результата от этого параметра при фиксированных прочих для аналитической и имитационной модели.

Предполагаемая структура такой программы показана на рисунке~\ref{fig:demoStruct}.

\begin{figure}[ht]
  \centering
  \includegraphics[width=\textwidth]{inc/dia/demoStruct}
  \caption{Структура демонстрационной программы}
  \label{fig:demoStruct}
\end{figure}

\section{Аналитическая модель системы}

Ниже представлен способ аналитического вычисления характеристик системы приведенной в предыдущем разделе.При выводе формул использовались методы укрупнения модели и укрупнения состояний описанные в~\cite{Kurov}.

\subsection{Моделирование отказов и восстановлений}
Состояние системы можно описать вектором $ \xi (t) = (\xi_{1}(t),\,\xi_{2}(t))$, где $\xi_{1}(t)$~--- число неисправных процессоров в момент времени $t$, $\xi_{2}(t)$~--- число неисправных каналов в момент времени $t$.

На рисунке~\ref{fig:broke-graph} показана структура фрагмента графа состояний системы, где $\beta_{ij}=\beta\frac{i}{i+j}min\left\lbrace i+j,L\right\rbrace$, $\delta_{ij}=\delta\frac{j}{i+j}min\left\lbrace i+j,L\right\rbrace$. 
\begin{figure}[ht]
\centering
\includegraphics[height=6cm]{inc/dia/broke-graph}
\caption{Структура фрагмента графа состояний системы}
\label{fig:broke-graph}
\end{figure}

Проведем укрупнение состояний системы. Объединим в одно макросостояние все вершины графа, у которых одинаковым является первый компонент $\xi_{1}(t)$~--- число неисправных процессоров. Полученный граф представлен на рисунке~\ref{fig:broke-proc}, где $\beta_i=\beta\sum\limits_{j=0}^N\pi_j\frac{i}{i+j}min\left\lbrace i+j,L\right\rbrace$, $\pi_j$~--- вероятность того, что отказали ровно j каналов.

\begin{figure}[ht]
\centering
\includegraphics[width=\textwidth]{inc/dia/broke-proc}
\caption{Граф состояний системы}
\label{fig:broke-proc}
\end{figure}

Тогда выражения для определения вероятностей стационарных состояний примут вид:

\begin{equation}
\label{eq:broke-proc}
\left\{
   \begin{array}{lcl}
	p_{0} = \left( 1 + \dfrac{M \alpha}{\beta_1} +  ... + \dfrac{M! \alpha^{M}}{\prod \limits_{i=1}^M \beta_i} \right) ^{-1} \\
	p_{i} = p_{0} \dfrac{\alpha^{i}\prod \limits_{j=1}^{i} (M-j+1)}{\prod \limits_{j=1}^i \beta_{j}}, \quad i = \overline{1,M}  \\ 
	\beta_i=\beta\sum\limits_{j=0}^N\pi_j\frac{i}{i+j}min\left\lbrace i+j,L\right\rbrace
   \end{array}
\right.
\end{equation}
 
Аналогичным образом объединим в одно макросостояние все вершины графа, у которых одинаковым является второй компонент $\xi_{2}(t)$~--- число неисправных каналов. Полученный граф представлен на рисунке~\ref{fig:broke-chan}, а выражения для определения вероятностей стационарных состояний примут вид:

\begin{equation}
\label{eq:broke-chan}
\left\{
   \begin{array}{lcl}
	\pi_{0} = \left( 1 + \dfrac{N \gamma}{\delta_1} +  ... + \dfrac{N! \gamma^{N}}{\prod \limits_{i=1}^N \delta_i} \right) ^{-1} \\
	\pi_{i} = \pi_{0} \dfrac{\gamma^{i}\prod \limits_{j=1}^{i} (N-j+1)}{\prod \limits_{j=1}^i \delta_{j}}, \quad i = \overline{1,N}  \\ 
	\delta_j=\delta\sum\limits_{i=0}^M p_i\frac{j}{i+j}min\left\lbrace i+j,L\right\rbrace
   \end{array}
\right.
\end{equation}

\hfill

\begin{figure}[ht]
\centering
\includegraphics[width=\textwidth]{inc/dia/broke-chan}
\caption{Граф состояний системы}
\label{fig:broke-chan}
\end{figure}

\hfill

Применяя формулы~\ref{eq:broke-proc} и~\ref{eq:broke-chan} итеративно получим вероятности отказов процессоров и каналов в системе. В качестве начального приближения можно взять $\pi_i=\frac{1}{M}$

\hfill

\subsection{Укрупнение модели}

\hfill

Заменим исходную модель агрегированной однофазной моделью АМ1 (см. рисунок~\ref{fig:AM1}). В агрегированный узел объединена подсистема, включающая в себя процессоры и каналы. Интенсивность обслуживания в этом узле зависит от числа находящихся в нем заявок.

\hfill

\begin{figure}[ht]
\centering
\includegraphics[height=7cm]{inc/dia/AM1}
\caption{Укрупненная модель АМ1}
\label{fig:AM1}
\end{figure}

\hfill

Граф состояний полученной системы представлен на рисунке~\ref{fig:graphAM1}. Производительность системы может быть вычислена по формулам:


\begin{equation}
\label{eq:AM1}
\left\{
   \begin{array}{lcl}
	\hat{\pi}_{0} = \left( 1 + \dfrac{K \lambda}{\xi_1} +  ... + \dfrac{K! \lambda^{K}}{\prod \limits_{i=1}^K \xi_i} \right) ^{-1} \\
	\hat{\pi}_{i} = \hat{\pi}_{0} \dfrac{\lambda^{i}\prod \limits_{j=1}^{i} (K-j+1)}{\prod \limits_{j=1}^i \xi_{j}}, \quad i = \overline{1,K}  \\ 
	\xi_{ср}^{*}=\sum \limits_{i=1}^K \xi_i \hat{\pi_i}
   \end{array}
\right.
\end{equation}


\begin{figure}[ht]
\centering
\includegraphics[width=\textwidth]{inc/dia/graphAM1}
\caption{Граф состояний модели АМ1}
\label{fig:graphAM1}
\end{figure}

Однако, чтобы воспользоваться приведенными формулами, необходимо знать параметры связи $\mu_i$. Чтобы их найти, рассмотрим укрупненную модель АМ2, структура и граф состояний которой показаны на рисунках~\ref{fig:AM2} и~\ref{fig:graphAM2}. За состояние системы примем количество заявок на процессорной фазе, а интенсивности переходов могут быть выражены по формулам:

\begin{equation}
\label{eq:mu}
\mu_i = \mu \sum \limits_{j=0}^M p_j min \left\lbrace i, M-j \right\rbrace
\end{equation}

\begin{equation}
\label{eq:nu}
\nu_i = \nu \sum \limits_{j=0}^N \pi_j min \left\lbrace n-i+1, N-j \right\rbrace
\end{equation}



\begin{figure}[ht]
\centering
\includegraphics[height=7cm]{inc/dia/AM2}
\caption{Укрупненная модель АМ2}
\label{fig:AM2}
\end{figure}

\begin{figure}[ht]
\centering
\includegraphics[width=\textwidth]{inc/dia/graphAM2}
\caption{Граф состояний модели АМ2}
\label{fig:graphAM2}
\end{figure}


Параметр связи может вычислен по следующим формулам:


\begin{equation}
\label{eq:AM2}
\left\{
   \begin{array}{lcl}
	\hat{p}_{0} = \left( 1 + \dfrac{\nu_1}{\mu_1} +  ... + \dfrac{\prod \limits_{i=1}^n \nu_i}{\prod \limits_{i=1}^n \mu_i} \right) ^{-1} \\
	\hat{p}_{i} = \hat{p}_{0} \dfrac{\prod \limits_{j=1}^{i} (\nu_j)}{\prod \limits_{j=1}^i \mu_{j}}, \quad i = \overline{1,n}  \\ 
	\xi_n = \sum \limits_{i=1}^n \hat{p}_i \mu_i
   \end{array}
\right.
\end{equation}


\subsection{Окончательная расчетная схема}
Последовательность расчета производительности системы должна быть следующей (см. рисунок~\ref{fig:idefAnalit}):

\begin{enumerate}
\item По формулам~\ref{eq:broke-proc} и ~\ref{eq:broke-chan} вычислить $\pi_i, i=\overline{0,N}$ и $p_i, i=\overline{0,M} $.
\item По формулам~\ref{eq:mu},~\ref{eq:nu} и~\ref{eq:AM2} вычислить $\xi_n, n=\overline{1,K}$.
\item Вычислить $\xi^{*}$ по формулам~\ref{eq:AM1}.
\end{enumerate}


\begin{figure}[ht!]
\centering
\includegraphics[width=0.9\textheight,angle=-90]{inc/dia/idefAnalit}
\caption{Последовательность расчета производительности системы}
\label{fig:idefAnalit}
\end{figure}

\section{Имитационная модель}

Ниже представлена имитационная модель демонстрационной системы, реализованная при помощи разработанной системы моделирования.

\subsection{Моделирование многоканальных обслуживающих устройств}

Так как в исследуемой системе используются многоканальные обслуживающие устройства, каждый из каналов которых может выйти из строя независимо от других, не представляется возможным промоделировать их при помощи хранилищ (хранилища могут быть отключены только целиком и не поддерживают вытеснение транзактов). Поэтому придется каждый канал обслуживающего устройства моделировать отдельным многоканальным устройством, а выбор транзактом свободного канала осуществлять при помощи блока TRANSFER, работающего в режиме ALL.

Ниже приведен код функции, формирующий часть модели, ответственную за обслуживание заявки на одном из процессоров. В качестве параметров функция принимает номер процессора, интенсивность обработки и метку блока с которого начинается следующая фаза обслуживания.

\begin{verbatim}
proc i mu l =
    do seize ("proc" ++ show i)
       advance (1/mu, xpdis)
       release ("proc" ++ show i)
       transfer ((),l)
\end{verbatim}

\subsection{Моделирование отказов и восстановлений}

Для моделирования отказов и восстановления каждого из каналов используется отдельный транзакт, который сперва захватывает соответствующее устройство при помощи блока PREEMPT в режиме прерывания (после этого устройство становится недоступно для транзактов, моделирующих задачи), после этого пытается захватить ресурс хранилища, моделирующего ремонтные бригады, и после того через некоторое время освобождает устройство и ресурс (устройство восстановлено и снова доступно для обработки задач). Время, остававшееся до окончания обработки транзакту, вытесненному в момент поломки, сохраняется в  параметре транзакта, а сам транзакт перенаправляется на фазу дообслуживания.


Ниже приведен код функции, формирующей часть модели ответственную за отказы и восстановления одного из процессоров. В качестве параметров функция принимает номер процессора, интенсивности его отказов и восстановлений и метку блока, с которого начинается фаза дообслуживания.

\begin{verbatim}
breakReairProc i alpha beta l' = 
    do generate (0,0,0,1)
       l <- advance (1/alpha, xpdis)
       preempt ("proc" ++ show i, (), l', 1, RE)
       enter "repairers"
       advance (1/beta,xpdis)
       return' ("proc" ++ show i)
       leave "repairers"
       transfer ((),l)
\end{verbatim}

Функция формирующая фазу дообслуживания во всем подобна функции \Code{proc} за исключением того, что время обслуживания берется из параметра транзакта, а не определяется случайным образом.

\begin{verbatim}
procFinish i l =
    do seize ("proc" ++ show i)
       advance (Pr 1)
       release ("proc" ++ show i)
       transfer ((),l)
\end{verbatim}

\subsection {Общая модель системы}

Функция формирующая при помощи вышеописанных общую модель системы показана ниже. В качестве параметров она принимает все параметры системы (количество задач, процессоров, каналов и ремонтных бригад и интенсивности обработки на различных фазах и отказов и восстановлений).

\begin{verbatim}
model m n k l lambda mu nu alpha beta gamma delta = 
    do storage ("repairers", l)
       generate (0,0,0,1)
       advance 100
       terminate 1
       
       generate (0,0,0,k)
       userPhase <- advance (1/lambda, xpdis)
       seize "counter"
       release "counter"
       procStart <- transfer (All,userPhase +5, 
                              userPhase + 5 + (m-1)*4,4)
       chanStart <- transfer (All,userPhase +5+4*m, 
                              userPhase + 5 + m*4 + (n-1)*4,4)
       mapM_ (\i -> proc i mu chanStart) [1..m]
       ls <- mapM (\i -> chan i nu userPhase) [1..n]
       
       let l = fromIntegral $ last ls
       procFinishL <- transfer (All,l + 3,
                                l + 3 + (m-1)*4,4)
       chanFinishL <- transfer (All,l + 3 + 4*m,
                                l + 3 + 4*m + (n-1)*4,4)
       mapM_ (\i -> procFinish i chanStart) [1..m]
       mapM_ (\i -> chanFinish i userPhase) [1..n]
       
       when (alpha > 0) $ mapM_ (\i -> breakReairProc 
                                         i alpha beta procFinishL
                                ) [1..m]
       when (gamma > 0) $ mapM_ (\i -> breakReairChan 
                                         i gamma delta chanFinishL
                                ) [1..n]
\end{verbatim}

 Первым делом функции объявляется хранилище емкостью равной числу ремонтных бригад. Затем создается группа блоков ответственная за ограничение времени моделирования.

Последующие блоки отвечают за создание транзактов, моделирующих задачи пользователей и время их обдумывания пользователем. Далее следуют блоки TRANSFER предназначенные для передачи заявки на свободный в данный момент процессор или канал. Далее путем многократного вызова функций \Code{proc} и \Code{chan} формируется часть модели ответственная за обработку задач на процессорной и канальной фазах. 

Далее аналогичным образом, при помощи блоков TRANSFER и многократного вызова функций \Code{procFinish} и \Code{chanFinish} формируются фазы дообслуживания. 

И в заключение при помощи функций \Code{breakReairProc} и \Code{breakReairChan} формируются блоки моделирующие отказы и восстановления.

\section{Выводы}

В результате проектирования был разработан синтаксис и алгоритм описания моделей, а также алгоритмы и структуры данных необходимые для имитационного моделирования. Были спроектированы структуры библиотеки имитационного моделирования и демонстрационной программы, призванной проиллюстрировать ее работу. Были разработаны аналитическая и имитационная модели демонстрационной системы, описанной в предыдущем разделе.


%\include{40-impl}

\backmatter %% Здесь заканчивается нумерованная часть документа и начинаются ссылки и
            %% заключение

%\chapter{Заключение}

В результате проделанной работы был проведен обзор системы моделирования GPSS и решены следующие задачи:

\begin{itemize}
\item Обоснован выбор подмножества блоков GPSS, реализуемых в работе.
\item Разработан и реализован алгоритм формирования моделей.
\item Реализованы алгоритмы имитационного моделирования сформированных моделей.
\item Разработаны аналитическая и имитационная модель типовой системы.
\item Реализована библиотека имитационного моделирования на языке Haskell.
\item Реализована демонстрационная программа, позволяющая оценить возможности разработанной библиотеки и оценить точность выбранных методов.
\item Проведено сравнение результатов аналитического и имитационного моделирования типовой системы, что подтвердило корректность реализации разработанной библиотеки.


\end{itemize}


% % Список литературы при помощи BibTeX
% Юзать так:
%
% pdflatex rpz
% bibtex rpz
% pdflatex rpz

\bibliographystyle{gost780u}
\bibliography{rpz}

%%% Local Variables: 
%%% mode: latex
%%% TeX-master: "rpz"
%%% End: 


\appendix   % Тут идут приложения
	

\end{document}

%%% Local Variables:
%%% mode: latex
%%% TeX-master: t
%%% End:
