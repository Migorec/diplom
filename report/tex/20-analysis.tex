\chapter{Аналитический раздел}
\label{cha:analysis}

В данном разделе проводится обзор принципов функционирования и синтаксиса системы GPSS, а также производтся выбор блоков, которые следует реализовать в разрабатываемой системе моделирования.

\section{Краткий обзор GPSS}

GPSS стал одним из первых языков моделирования, облегчающих процесс написания имитационных программ. Он был создан в виде конечного продукта Джефри Гордоном в фирме IBM в 1962~г.\cite{ImitGPSS} В свое время он входил в десятку лучших языков программирования и по сей день широко используется для решения практических задач.

Основой имитационных алгоритмов GPSS является дискретно-событийный подход~--- моделирование сис­темы в дискретные моменты времени, когда происходят события, от­ражающие последовательность изменения состояний системы во времени.\cite{ImitGPSS}

\section{Объекты языка GPSS}

Основными Объектами языка GPSS являются транзакты и блоки, которые отображают соответственно динамические и статические объекты моделируемой системы.

Транзакты~--- динамические элементы GPSS-модели. В реальной системе транзактам могут соответствовать такие элементы как заявка, покупатель автомобиль и др. Состояни транзакта в процессе моделирования хараактеризуется следующими атрибутами:

\begin{itemize}
\item параметры~--- набор значений связанных с транзактом. Каждый транзакт может иметь произвольное число параметров. Каждый параметр имеет уникальный номер, по которому на него можно сослаться;
\item приоритет~--- определяет порядок продвижения транзактов при конкурировании за общий ресурс;
\item текущий блок~--- номер блока, в котором транзакт находится в данный момент;
\item следующий блок~--- номер блока, в который транзакт попыытается войти;
\item время появления транзакта~---  момент времени в который транзакт был создан;
\item состояние~--- состояние, показывающее в каких списках транзакт находится в даннный момент. Транзакт может находиться в одном из следующих состояний:
    \begin{enumerate}
    \item активен~--- транзакт находится в списке текущих событий и имеет наивысший приоритет;
    \item приостановлен~--- транзакт находится в списке будущих событий либо в списке текущих событий, но с меньшим приоритетом;
    \item пассивен~--- транзакт находится в списке прерываний, списке синхронизации, списке блокировок или списке пользователя;
    \item завершен~--- транзакт уничтожен и болше не участвует в модели.
    \end{enumerate}
    Диаграмма состояний транзакта показана на Рисунке~\ref{fig:transactionState}.
\end{itemize}


\begin{figure}[ht]
  \centering
  \includegraphics[width=\textwidth]{inc/dia/transactionState}
  \caption{Состояния транзакта}
  \label{fig:transactionState}
\end{figure}


Блоки~--- статические элементы GPSS-модели. Модель в GPSS может быть представленна как диаграмма блоков, т.е. ориентированный граф, узлами которого являются блоки, а дугам~--- направления движения транзактов. с каждым блоком связано некоторое действие, изменяющее состояние прочих элементов модели. Транзакты проходят блоки один за другим, до тех пор пока не достигнут блока TERMINATE. В ряде случаев транзакт может быть остановлен в одном из блоков до наступления некоторого события.


Помимо транзактов и блоков в GPSS используются следующие объекты: устройства, многоканальные устройства (хранилища, памяти), ключи, очереди, списки пользователя и др.


\section{Управления процесом моделирования в GPSS}

В системе GPSS итерпретатор поддерживает сложные структуры организации списков (см. Рисунок~\ref{fig:GPSSChains}).\cite{ImitGPSS} Два основных из них~--- список текущих событий (СТС) и список будущих событий (СБС).

\begin{figure}[ht]
  \centering
  \includegraphics[width=\textwidth]{inc/dia/gpss}
  \caption{Списки GPSS}
  \label{fig:GPSSChains}
\end{figure}

В СТС входят все события запланированные  на текущий  момент модельного времени. Интерпретатор в первую очередь просматривает этот список и перемещает по модели те транзакты, для которых выполнены все условия. Если таких транзактов в списке не оказалось интерпретатор обращается к СБС. Он переносит все события, запланированные на ближайший момент времени и вновь возвращается к просмотру СТС. Перенос также осуществляется в случае совпадения текущего момента времени с моментом наступления ближайшего события из СБС.

В целях эффективной организации просмотра транзактов, движение которых заблокировано (например, из-за занятости некоторого ресурса), используются следующие вспомогательные списки:

\begin{itemize}
\item списки блокировок~--- списки транзактов, которые ожидают освобождения некоторого ресурса;
\item список прерываний~--- содержит транзакты, прерванные во время обслуживания. Используется для организации обслуживания одноканальных устройств с абсолютным приоритетом;
\item списки синхронизации~--- содержат транзакты одного семейства (созданные блоком SPLIT), которые ожидают синхронизации в блоках (MATCH, ASSEMBLE или GATHER);
\item списки пользователя~--- содержат транзакты, выведенные пользователем из СТС с помощью блока LINK. Транзакты могут быть возвращены в СТСс помощью блока UNLINK.
\end{itemize}




\section{Выбор подмножества реализуемых блоков}

В современной версии языка GPSS (входящей в пакет GPSS World) поддерживается 53 различных блока.\cite{GPSSRef} В рамках данной работы не представляется возможным реализовать  аналоги каждого из них. Поэтому следует выделить некоторое подмножество блоков, которое с одной не будет слишком обширным, а с другой~--- позволит решать практические или по крайней мере учебные задачи.

В качетсве примера рассмотрим задачу из курса Модели оценки качества аппаратно программных комлексов:

\begin{quote}
В вычислительной системе, содержащей N процессоров и M каналов обмена данными, постоянно находятся K задач. Разработать модель, оценивающую производительность системы с учетом отказов и восстановлений процессоров и каналов. Имеется не более L ремонтных бригад, которые ремонтируют отказывающие устройства с бесприоритетной  дисциплиной. Интенсивность отказов, восстановлений, средние времена обработки сообщения и среднее время обдумывания также известны.
\end{quote}

Схема модели данной системы показана на Рисунке~\ref{fig:mainModel}

\begin{figure}[ht]
\centering
\includegraphics[width=\textwidth]{inc/dia/main}
\caption{Схема моделируемой системы}
\label{fig:mainModel}
\end{figure}

Как и подаляющее большинство других задач, данная задача, безусловно, не может быть решена без использования блоков GENERATE, TERMINATE и ADVANCE. Так как моделируемая система является замкнутой, при описании модели не обойтись без блока TRANSFER.

К сожалению, не представляется возможным реализовать процессоры и каналы как многоканальные устройства, т.к. многоканальные устройства в GPSS не поддерживают абсолютные приоритеты и не позволяют смоделировать выход из строя отдельных каналов устройства. Однако, требуемую систему можно описать при помощи множества одноканальных устройств и блока TRANSFER в режиме ALL. Таким образом, также понадобятся блоки SEIZE и RELEASE. Для моделирования отказов устройств можно воспользоваться блоками SAVAIL и SUNAVAIL либо блоками PREEMPT и RETURN.

Наконец, доступные ремонтные бригады можно смоделировать с помощью многоканального устройства. Соответственно, понадобятся блоки ENTER и LEAVE.

Приблизительная модель системы показана в Листинге~\ref{lst:sample01}

\lstinputlisting[caption=Приблизительная модель системы,label=lst:sample01]{inc/src/analysModel.gpss}

Таким образом, разрабатываемая система имитационного моделирования должна поддерживать аналоги по крайней мере следующих блоков: ADVANCE, ENTER, GENERATE, LEAVE, PREEMPT, RELEASE, RETURN, SEIZE, TERMINATE и TRANSFER.

\section{Описание выбранных блоков}

Ниже представлено описание выбранных блоков в соответствии со справочным руководством GPSS World.\cite{GPSSRef}

\subsection*{ADVANCE A,B}

Блок ADVANCE осуществляет задержку продвижения транзактов на заданный промежуток времени.

A~--- Среднее время задержки. Не обязательный параметр. Значение по умолчанию~---~0.

B~--- Максимально допустимое отклонение времени задержки либо функция-модификатор.

\subsection*{ENTER A,B}

При входе в блок ENTER транзакт либо занимает заданное колличество каналов указанного многоканального устройства либо блокируется до его освобождения.

A~--- Имя или номер многоканального устройства. Обязательный параметр.

B~--- Число требуемых каналов. Не обязательный параметр. Значение по умолчанию~--- 1.

\subsection*{GENERATE A,B,C,D,E}

Блок GENERATE предназначен для создания новых транзактов.

A~--- Среднее время между генерацией последовательных заявок. Не обязательный параметр.

B~--- Максимальное допустимое отклонение времени генерации либо функция-модификатор. Не обязательный параметр.

С~--- Задержка до начала генерации первого транзакта. Не обязательный параметр.

D~--- Ограничение на максимальное допустимое число созданных транзактов. Не обязательный параметр. Пол умолчанию ограничение отсутствует.

E~--- Уровень приоритета создаваемых заявок. Не обязательный параметр. Значение по умолчанию~--- 0.

\subsection*{LEAVE A,B}

При входе в блок LEAVE транзакт освобождаает заданное число каналов указанного многоканального устройства.

A~--- Имя или номер многоканального устройства. Обязательный параметр.

B~--- Число требуемых каналов. Не обязательный параметр. Значение по умолчанию~--- 1.

\subsection*{PREEMPT A,B,C,D,E}

Блок PREEMPT подобен блоку SEIZE и вошедший в него транзакт также пытается занять указанное одноканальное устройство. Однако, в случае если устройство занято, а приоритет поступившего транзакта выше, чем у обслуживающегося в данный момент, обслуживающийся транзакт вытесняется с устройства. Его дальнейшее поведение определяется параметрами блока PREEMPT.

\subsection*{RELEASE A}

Блок RELEASE освобождает одноканальное устройство.

A~--- Имя или номер одноканального устройства. Обязательный параметр.

\subsection*{RETURN A}

Блок RELEASE освобождает одноканальное устройство.

A~--- Имя или номер одноканального устройства. Обязательный параметр.

\subsection*{SEIZE A}

При входе в блок SEIZE транзакт занимает указанное одноканальное устройство либо блокируется до его освобождения.

A~--- Имя или номер одноканального устройства. Обязательный параметр.

\subsection*{TERMINATE A}

Блок TERMINATE завершает поступивший в него транзакт. И опционально уменьшает счетчик завершенных транзаков. Когда счетчик достигает нуля имитация останавливается.

A~--- Значение, на которое следует уменьшить счетчик завершенных транзактов. Не обязательный параметр. Значение по умолчанию~--- 0.

\section{Аналитическая модель системы}

Для того, чтобы протестировать разработанную систему модилирования, целесообразно (помимо прочего) разработать аналитическую модель, позволяющую вычислить искомые характеристики системы, с целью сравнить их с величинами, полученными при помощи имитационного моделирования. Для этого были использованы приемы и методы описанные в~\cite{Kurov}.

\subsection{Моделирование отказов и восстановлений}
Состояние системы можно описать вектором $ \xi (t) = (\xi_{1}(t),\,\xi_{2}(t))$, где $\xi_{1}(t)$~--- число неисправных процессоров в момент времени $t$, $\xi_{2}(t)$~--- число неисправных каналов в момент времени $t$.

На рисунке~\ref{fig:broke-graph} показана структура фрагмента графа состояний системы, где $\beta_{ij}=\beta\frac{i}{i+j}min\left\lbrace i+j,L\right\rbrace$, $\delta_{ij}=\delta\frac{j}{i+j}min\left\lbrace i+j,L\right\rbrace$. 
\begin{figure}[ht]
\centering
\includegraphics[height=6cm]{inc/dia/broke-graph}
\caption{Структура фрагмента графа состояний системы}
\label{fig:broke-graph}
\end{figure}

Проведем укрупнение состояний сисетмы. Объединим в одно макросостояние все врешины графа, у которых одинаковым является первый компонент $\xi_{1}(t)$~--- число неисправных процессоров. Полученный граф представлен на рисунке~\ref{fig:broke-proc}, где $\beta_i=\beta\sum\limits_{j=0}^N\pi_j\frac{i}{i+j}min\left\lbrace i+j,L\right\rbrace$, $\pi_j$~--- вероятность того, что отказали ровно j каналов.

\begin{figure}[ht]
\centering
\includegraphics[width=\textwidth]{inc/dia/broke-proc}
\caption{Граф состояний системы}
\label{fig:broke-proc}
\end{figure}

Тогда выражения для определения вероятностей стационарных состояний примут вид:

\begin{equation}
\label{eq:broke-proc}
\left\{
   \begin{array}{lcl}
	p_{0} = \left( 1 + \dfrac{M \alpha}{\beta_1} +  ... + \dfrac{M! \alpha^{M}}{\prod \limits_{i=1}^M \beta_i} \right) ^{-1} \\
	p_{i} = p_{0} \dfrac{\alpha^{i}\prod \limits_{j=1}^{i} (M-j+1)}{\prod \limits_{j=1}^i \beta_{j}}, \quad i = \overline{1,M}  \\ 
	\beta_i=\beta\sum\limits_{j=0}^N\pi_j\frac{i}{i+j}min\left\lbrace i+j,L\right\rbrace
   \end{array}
\right.
\end{equation}
 
Аналогичным образом объединим в одно макросостояние все врешины графа, у которых одинаковым является второй компонент $\xi_{2}(t)$~--- число неисправных каналов. Полученный граф представлен на рисунке~\ref{fig:broke-chan}, а выражения для определения вероятностей стационарных состояний примут вид:

\begin{equation}
\label{eq:broke-chan}
\left\{
   \begin{array}{lcl}
	\pi_{0} = \left( 1 + \dfrac{N \gamma}{\delta_1} +  ... + \dfrac{N! \gamma^{N}}{\prod \limits_{i=1}^N \delta_i} \right) ^{-1} \\
	\pi_{i} = \pi_{0} \dfrac{\gamma^{i}\prod \limits_{j=1}^{i} (N-j+1)}{\prod \limits_{j=1}^i \delta_{j}}, \quad i = \overline{1,N}  \\ 
	\delta_j=\delta\sum\limits_{i=0}^M p_i\frac{j}{i+j}min\left\lbrace i+j,L\right\rbrace
   \end{array}
\right.
\end{equation}

\hfill

\begin{figure}[ht]
\centering
\includegraphics[width=\textwidth]{inc/dia/broke-chan}
\caption{Граф состояний системы}
\label{fig:broke-chan}
\end{figure}

\hfill

Применяя формулы~\ref{eq:broke-proc} и~\ref{eq:broke-chan} итеративно получим вероятности отказов процессоров и каналов в системе. В качестве начального приближения можно взять $\pi_i=\frac{1}{M}$

\hfill

\subsection{Укрупнение модели}

\hfill

Заменим исходную модель агрегированной однофазной моделью АМ1 (см. рисунок~\ref{fig:AM1}). В агрегированный узел объединена подсистема, включающая в себя процессоры и каналы. Интенсивность обслуживания в этом узле зависит от числа находящихся в нем заявок.

\hfill

\begin{figure}[ht]
\centering
\includegraphics[height=7cm]{inc/dia/AM1}
\caption{Укрупненная модель АМ1}
\label{fig:AM1}
\end{figure}

\hfill

Граф состояний полученной системы представлен на рисунке~\ref{fig:graphAM1}. Производительность системы может быть вычислена по формулам:


\begin{equation}
\label{eq:AM1}
\left\{
   \begin{array}{lcl}
	\hat{\pi}_{0} = \left( 1 + \dfrac{K \lambda}{\xi_1} +  ... + \dfrac{K! \lambda^{K}}{\prod \limits_{i=1}^K \xi_i} \right) ^{-1} \\
	\hat{\pi}_{i} = \hat{\pi}_{0} \dfrac{\lambda^{i}\prod \limits_{j=1}^{i} (K-j+1)}{\prod \limits_{j=1}^i \xi_{j}}, \quad i = \overline{1,K}  \\ 
	\xi_{ср}^{*}=\sum \limits_{i=1}^K \xi_i \hat{\pi_i}
   \end{array}
\right.
\end{equation}


\begin{figure}[ht]
\centering
\includegraphics[width=\textwidth]{inc/dia/graphAM1}
\caption{Граф состояний модели АМ1}
\label{fig:graphAM1}
\end{figure}

Однако, чтобы воспользоваться приведенными формулами, необходимо знать параметры связи $\mu_i$. Чтобы их найти, рассмотрим укрупненную модель АМ2, структура и граф состояний которой показаны на рисунках~\ref{fig:AM2} и~\ref{fig:graphAM2}. За состояние системы примем количество заявок на процессорной фазе, а интенсивности переходов могут быть выражены по формулам:

\begin{equation}
\label{eq:mu}
\mu_i = \mu \sum \limits_{j=0}^M p_j min \left\lbrace i, M-j \right\rbrace
\end{equation}

\begin{equation}
\label{eq:nu}
\nu_i = \nu \sum \limits_{j=0}^N \pi_j min \left\lbrace n-i+1, N-j \right\rbrace
\end{equation}



\begin{figure}[ht]
\centering
\includegraphics[height=7cm]{inc/dia/AM2}
\caption{Укрупненная модель АМ2}
\label{fig:AM2}
\end{figure}

\begin{figure}[ht]
\centering
\includegraphics[width=\textwidth]{inc/dia/graphAM2}
\caption{Граф состояний модели АМ2}
\label{fig:graphAM2}
\end{figure}


Параметр связи может вычислен по следующим формулам:


\begin{equation}
\label{eq:AM2}
\left\{
   \begin{array}{lcl}
	\hat{p}_{0} = \left( 1 + \dfrac{\nu_1}{\mu_1} +  ... + \dfrac{\prod \limits_{i=1}^n \nu_i}{\prod \limits_{i=1}^n \mu_i} \right) ^{-1} \\
	\hat{p}_{i} = \hat{p}_{0} \dfrac{\prod \limits_{j=1}^{i} (\nu_j)}{\prod \limits_{j=1}^i \mu_{j}}, \quad i = \overline{1,n}  \\ 
	\xi_n = \sum \limits_{i=1}^n \hat{p}_i \mu_i
   \end{array}
\right.
\end{equation}


\subsection{Окончательная расчетная схема}
Последовательность расчета производительности системы должна быть следующей:

\begin{enumerate}
\item По формулам~\ref{eq:broke-proc} и ~\ref{eq:broke-chan} вычислить $\pi_i, i=\overline{0,N}$ и $p_i, i=\overline{0,M} $.
\item По формулам~\ref{eq:mu},~\ref{eq:nu} и~\ref{eq:AM2} вычислить $\xi_n, n=\overline{1,K}$.
\item Вычислить $\xi^{*}$ по формулам~\ref{eq:AM1}.
\end{enumerate}

\section{Выводы}

Был проведен анализ устройства системы GPSS и осуществлен выбор подмножества блоков, необходимых для моделирования не сложных систем массового обслуживания. Также была построена аналитическая модель такой системы, с целью использовать ее как эталон при тестировании разрабатыываемой системы моделирования.
