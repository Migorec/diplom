\chapter{Аналитический раздел}
\label{cha:analysis}

В данном разделе проводится обзор принципов функционирования и синтаксиса системы GPSS, а также производтся выбор блоков, которые следует реализовать в разрабатываемой системе моделирования.

\section{Краткий обзор GPSS}

GPSS стал одним из первых языков моделирования, облегчающих процесс написания имитационных программ. Он был создан в виде конечного продукта Джефри Гордоном в фирме IBM в 1962~г.\cite{ImitGPSS} В свое время он входил в десятку лучших языков программирования и по сей день широко используется для решения практических задач.

Основой имитационных алгоритмов GPSS является дискретно-событийный подход~--- моделирование сис­темы в дискретные моменты времени, когда происходят события, от­ражающие последовательность изменения состояний системы во времени.\cite{ImitGPSS}

\section{Элементы языка GPSS}

Основными элементами языка GPSS являются транзакты и блоки, которые отображают соответственно динамические и статические объекты моделируемой системы.

Транзакты~--- динамические элементы GPSS-модели. В реальной системе транзактам могут соответствовать такие элементы как заявка, покупатель автомобиль и др. Состояни транзакта в процессе моделирования хараактеризуется следующими атрибутами:

\begin{itemize}
\item параметры~--- набор значений связанных с транзактом. Каждый транзакт может иметь произвольное число параметров. Каждый параметр иметт уникальный номер, по которому на него можно сослаться;
\item приоритет~--- определяет порядок продвижения транзактов при конкурировании за общий ресурс;
\item текущий блок~--- номер блока, в котором транзакт находится в данный момент;
\item следующий блок~--- номер блока, в который транзакт попыытается войти;
\item время появления блока~---  момент времени в который транзакт был создан;
\item состояние~--- состояние, показывающее в каких списках транзакт находится в даннный момент. Транзакт может находиться в одном из следующих состояний:
    \begin{enumerate}
    \item активен~--- транзакт находится в списке текущих событий и имеет наивысший приоритет;
    \item приостановлен~--- транзакт находится в списке будущих событий либо в списке текущих событий, но с меньшим приоритетом;
    \item пассивен~--- транзакт находится в списке прерываний, списке синхронизации, списке блокировок или списке пользователя;
    \item завершен~--- транзакт уничтожен и болше не участвует в модели.
    \end{enumerate}
    Диаграмма состояний транзакта показана на Рисунке~\ref{fig:transactionState}.
\end{itemize}


\begin{figure}[ht]
  \centering
  \includegraphics[width=\textwidth]{inc/dia/transactionState}
  \caption{Состояния транзакта}
  \label{fig:transactionState}
\end{figure}


Блоки~--- статические элементы GPSS-модели. Модель в GPSS может быть представленна как диаграмма блоков, т.е. ориентированный граф, узлами которого являются блоки, а дугам~--- направления движения транзактов.

\begin{figure}[ht]
  \centering
  \includegraphics[width=\textwidth]{inc/dia/gpss}
  \caption{Списки GPSS}
  \label{fig:GPSSChains}
\end{figure}



\section{Выбор подмножества блоков}


