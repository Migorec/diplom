\chapter{Аналитический раздел}
\label{cha:analysis}

В данном разделе проводится обзор принципов функционирования и синтаксиса системы GPSS, а также производится выбор подмножества возможностей GPSS, которые следует реализовать в разрабатываемой системе.

\section{Краткий обзор GPSS}

GPSS стал одним из первых языков моделирования, облегчающих процесс написания имитационных программ. Он был создан в виде конечного продукта Джефри Гордоном в фирме IBM в 1962~г.\cite{ImitGPSS} В свое время он входил в десятку лучших языков программирования и по сей день широко используется для решения практических задач. Наиболее современной версией GPSS для персональных компьютеров на данный момент является пакет GPSS World, разработанный компанией Minuteman Software.

Описание системы на GPSS представляет собой последовательность блоков,
каждый из которых соответствует некоторому оператору (подпрограмме). Каждый блок
имеет определенное количество параметров (полей).


Основой имитационных алгоритмов GPSS является дискретно-событийный подход~--- моделирование системы в дискретные моменты времени, когда происходят события, отражающие последовательность изменения состояний системы во времени.\cite{ImitGPSS}

\section{Объекты языка GPSS}

Основными Объектами языка GPSS являются транзакты и блоки, которые отображают соответственно динамические и статические объекты моделируемой системы.

Транзакты~--- динамические элементы GPSS-модели. В реальной системе транзактам могут соответствовать такие элементы как заявка, покупатель автомобиль и др. Состояние транзакта в процессе моделирования характеризуется следующими атрибутами:

\begin{enumerate}
\item параметры~--- набор значений связанных с транзактом. Каждый транзакт может иметь произвольное число параметров. Каждый параметр имеет уникальный номер, по которому на него можно сослаться;
\item приоритет~--- определяет порядок продвижения транзактов при конкурировании за общий ресурс;
\item текущий блок~--- номер блока, в котором транзакт находится в данный момент;
\item следующий блок~--- номер блока, в который транзакт попытается войти;
\item время появления транзакта~---  момент времени в который транзакт был создан;
\item состояние~--- состояние, показывающее в каких списках транзакт находится в данный момент. Транзакт может находиться в одном из следующих состояний:
    \begin{enumerate}
    \item активен~--- транзакт находится в списке текущих событий и имеет наивысший приоритет;
    \item приостановлен~--- транзакт находится в списке будущих событий либо в списке текущих событий, но с меньшим приоритетом;
    \item пассивен~--- транзакт находится в списке прерываний, списке синхронизации, списке блокировок или списке пользователя;
    \item завершен~--- транзакт уничтожен и больше не участвует в модели.
    \end{enumerate}
    Диаграмма состояний транзакта показана на Рисунке~\ref{fig:transactionState}.
\end{enumerate}


\begin{figure}[ht]
  \centering
  \includegraphics[width=\textwidth]{inc/dia/transactionState}
  \caption{Состояния транзакта}
  \label{fig:transactionState}
\end{figure}


Блоки~--- статические элементы GPSS-модели. Модель в GPSS может быть представлена как диаграмма блоков, т.е. ориентированный граф, узлами которого являются блоки, а дугам~--- направления движения транзактов. с каждым блоком связано некоторое действие, изменяющее состояние прочих элементов модели. Транзакты проходят блоки один за другим, до тех пор пока не достигнут блока TERMINATE. В ряде случаев транзакт может быть остановлен в одном из блоков до наступления некоторого события.


Помимо транзактов и блоков в GPSS используются следующие объекты: устройства, многоканальные устройства (хранилища, памяти), ключи, очереди, списки пользователя и др.


\section{Управления процессом моделирования в GPSS}

В системе GPSS интерпретатор поддерживает сложные структуры организации списков (см. Рисунок~\ref{fig:GPSSChains}).\cite{ImitGPSS} Два основных из них~--- список текущих событий (СТС) и список будущих событий (СБС).

\begin{figure}[ht]
  \centering
  \includegraphics[width=\textwidth]{inc/dia/gpss}
  \caption{Списки GPSS}
  \label{fig:GPSSChains}
\end{figure}

В СТС входят все события запланированные  на текущий  момент модельного времени. Интерпретатор в первую очередь просматривает этот список и перемещает по модели те транзакты, для которых выполнены все условия. Если таких транзактов в списке не оказалось интерпретатор обращается к СБС. Он переносит все события, запланированные на ближайший момент времени и вновь возвращается к просмотру СТС. Перенос также осуществляется в случае совпадения текущего момента времени с моментом наступления ближайшего события из СБС.

В целях эффективной организации просмотра транзактов, движение которых заблокировано (например, из-за занятости некоторого ресурса), используются следующие вспомогательные списки:

\begin{itemize}
\item списки блокировок~--- списки транзактов, которые ожидают освобождения некоторого ресурса;
\item список прерываний~--- содержит транзакты, прерванные во время обслуживания. Используется для организации обслуживания одноканальных устройств с абсолютным приоритетом;
\item списки синхронизации~--- содержат транзакты одного семейства (созданные блоком SPLIT), которые ожидают синхронизации в блоках (MATCH, ASSEMBLE или GATHER);
\item списки пользователя~--- содержат транзакты, выведенные пользователем из СТС с помощью блока LINK. Транзакты могут быть возвращены в СТС помощью блока UNLINK.
\end{itemize}




\section{Выбор подмножества реализуемых блоков}

В современной версии языка GPSS (входящей в пакет GPSS World) поддерживается 53 различных блока.\cite{GPSSRef} В рамках данной работы не представляется возможным реализовать  аналоги каждого из них. Поэтому следует выделить некоторое подмножество блоков, которое с одной не будет слишком обширным, а с другой~--- позволит решать практические или по крайней мере учебные задачи.

В качестве примера рассмотрим задачу из курса Модели оценки качества аппаратно программных комплексов:

\begin{quote}
В вычислительной системе, содержащей N процессоров и M каналов обмена данными, постоянно находятся K задач. Разработать модель, оценивающую производительность системы с учетом отказов и восстановлений процессоров и каналов. Имеется не более L ремонтных бригад, которые ремонтируют отказывающие устройства с бесприоритетной  дисциплиной. Интенсивность отказов, восстановлений, средние времена обработки сообщения и среднее время обдумывания также известны.
\end{quote}

Схема модели данной системы показана на Рисунке~\ref{fig:mainModel}

\begin{figure}[ht]
\centering
\includegraphics[width=\textwidth]{inc/dia/main}
\caption{Схема моделируемой системы}
\label{fig:mainModel}
\end{figure}

Как и подавляющее большинство других задач, данная задача, безусловно, не может быть решена без использования блоков GENERATE, TERMINATE и ADVANCE. Так как моделируемая система является замкнутой, при описании модели не обойтись без блока TRANSFER.

К сожалению, не представляется возможным реализовать процессоры и каналы как многоканальные устройства, т.к. многоканальные устройства в GPSS не поддерживают абсолютные приоритеты и не позволяют смоделировать выход из строя отдельных каналов устройства. Однако, требуемую систему можно описать при помощи множества одноканальных устройств и блока TRANSFER в режиме ALL. Таким образом, также понадобятся блоки SEIZE и RELEASE. Для моделирования отказов устройств можно воспользоваться блоками FAVAIL и FUNAVAIL либо блоками PREEMPT и RETURN.

Наконец, доступные ремонтные бригады можно смоделировать с помощью многоканального устройства. Соответственно, понадобятся блоки ENTER и LEAVE.

Приблизительная модель системы показана в Листинге~\ref{lst:sample01}

\lstinputlisting[caption=Приблизительная модель системы,label=lst:sample01]{inc/src/analysModel.gpss}

Таким образом, разрабатываемая система имитационного моделирования должна поддерживать аналоги по крайней мере следующих блоков: ADVANCE, ENTER, GENERATE, LEAVE, PREEMPT, RELEASE, RETURN, SEIZE, TERMINATE и TRANSFER.

\section{Описание выбранных блоков}

Ниже представлено описание выбранных блоков в соответствии со справочным руководством GPSS World.\cite{GPSSRef}

\subsection*{ADVANCE A,B}

Блок ADVANCE осуществляет задержку продвижения транзактов на заданный промежуток времени.

A~--- Среднее время задержки. Не обязательный параметр. Значение по умолчанию~---~0.

B~--- Максимально допустимое отклонение времени задержки либо функция-модификатор.

\subsection*{ENTER A,B}

При входе в блок ENTER транзакт либо занимает заданное количество каналов указанного многоканального устройства либо блокируется до его освобождения.

A~--- Имя или номер многоканального устройства. Обязательный параметр.

B~--- Число требуемых каналов. Не обязательный параметр. Значение по умолчанию~--- 1.

\subsection*{GENERATE A,B,C,D,E}

Блок GENERATE предназначен для создания новых транзактов.

A~--- Среднее время между генерацией последовательных заявок. Не обязательный параметр.

B~--- Максимальное допустимое отклонение времени генерации либо функция-модификатор. Не обязательный параметр.

С~--- Задержка до начала генерации первого транзакта. Не обязательный параметр.

D~--- Ограничение на максимальное допустимое число созданных транзактов. Не обязательный параметр. Пол умолчанию ограничение отсутствует.

E~--- Уровень приоритета создаваемых заявок. Не обязательный параметр. Значение по умолчанию~--- 0.

\subsection*{LEAVE A,B}

При входе в блок LEAVE транзакт освобождает заданное число каналов указанного многоканального устройства.

A~--- Имя или номер многоканального устройства. Обязательный параметр.

B~--- Число требуемых каналов. Не обязательный параметр. Значение по умолчанию~--- 1.

\subsection*{PREEMPT A,B,C,D,E}

Блок PREEMPT подобен блоку SEIZE и вошедший в него транзакт также пытается занять указанное одноканальное устройство. Однако, данный блок позволяет транзакту занять устройство, даже если в данный момент оно уже занято другим транзактом, при соблюдении ряда условий, определяемых параметрами блока.

A~--- Имя или номер одноканального устройства. Обязательный параметр.

B~--- задает режим работы блока. PR~--- режим приоритетов. По умолчанию~--- режим прерываний. В режиме прерываний транзакт может вытеснить из устройства любой другой транзакт, если тот в свою очередь не захватил устройство через блок PREEMPT. В режиме приоритетов транзакт может вытеснить любой транзакт с меньшим приоритетом.

C~--- задает номер блока, куда будет направлен транзакт вытесненный и устройства в результате действия блока PREEMPT. 

D~--- задает номер параметра вытесненного транзакта, в  котором будет сохранено время, которое осталось транзакту до окончания обработки а устройстве.

E~--- задает режим удаления вытесненного транзакта. RE~--- вытесненное сообщение удаляется из устройства и более не претендует на владение им. Требует обязательного указания параметра C. Значение по умолчанию~--- вытесненный транзакт будет вновь пытаться занять устройство. 

\subsection*{RELEASE A}

Блок RELEASE освобождает одноканальное устройство.

A~--- Имя или номер одноканального устройства. Обязательный параметр.

\subsection*{RETURN A}

Блок RELEASE освобождает одноканальное устройство.

A~--- Имя или номер одноканального устройства. Обязательный параметр.

\subsection*{SEIZE A}

При входе в блок SEIZE транзакт занимает указанное одноканальное устройство либо блокируется до его освобождения.

A~--- Имя или номер одноканального устройства. Обязательный параметр.

\subsection*{TERMINATE A}

Блок TERMINATE завершает поступивший в него транзакт. И опционально уменьшает счетчик завершенных транзактов. Когда счетчик достигает нуля имитация останавливается.

A~--- Значение, на которое следует уменьшить счетчик завершенных транзактов. Не обязательный параметр. Значение по умолчанию~--- 0.

\subsection*{TRANSFER A,B,C,D}

Блок TRANSFER является основным средством позволяющим изменить маршрут транзакта и перенаправить его к произвольному блоку модели. Параметр A определяет режим работы блока. Смысл остальных параметров меняется в зависимости от выбранного режима.

\paragraph{Безусловный режим}. Если параметр А пропущен, то блок TRANSFER работает в безусловном режиме. Входящий в блок TRANSFER транзакт переходит к блоку, указанному в поле В.

\paragraph{Статистический режим}. Параметр А является числом от 0 до 1, показывающим какая доля транзактов перейдет к блоку, указанному в параметре С. Остальные транзакты переходят к блоку,ь указанному в параметре B.

\paragraph{Режим BOTH}. Если параметр А равен BOTH, то блок TRANSFER работает в одноименном режиме.В этом режиме каждый входящий транзакт сначала пытается перейти к блоку, указанному в поле В. Если это сделать не удается, транзакт пытается перейти к блоку, указанному в поле С. Если транзакт не сможет перейти ни к тому, ни к другому блоку, он остается в блоке TRANSFER и будет повторять в том порядке попытки перехода при каждом просмотре списка текущих событий, до тех пор, пока не сможет выйти из блока TRANSFER.

\paragraph{Режим ALL}. Если параметр А равен ALL, то блок TRANSFER работает в одноименном режиме.В этом режиме каждый входящий транзакт1 прежде всего пытается перейти к блоку, указанному в поле В. Если транзакт не может войти в этот блок, то последовательно проверяются все блоки в определенном ряду в поисках первого, способного принять это сообщение, включая последний блок, указанный операндом С. Номер каждого проверяемого блока вычисляется как сумма номера предыдущего блока и шага, заданного операндом D. По умолчанию значение операнда D принимается равным 1.

\paragraph{Режим PICK}. Если параметр А равен PICK, то блок TRANSFER работает в одноименном режиме. Этот режим подобен режиму ALL, за тем исключением, что блок назначения выбирается случайным образом с одинаковой вероятностью.

\section{Выводы}

Был проведен обзор устройства системы GPSS и осуществлен выбор подмножества блоков, необходимых для моделирования не сложных систем массового обслуживания. Представлено описание назначения и параметров каждого из выбранных блоков.
