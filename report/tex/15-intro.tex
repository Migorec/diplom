\Introduction

Зародившаяся в начале прошлого века с целью упорядочить работу телефонных станций, теория массового обслуживания нашла применения в моделировании самых разнообразных систем, таких как системы связи, обработки информации, снабжения, производства и др.

Несмотря на имеющиеся достижения в области математического исследования характеристик систем массового обслуживания, наиболее универсальным подходом по прежнему остается имитационное моделирование.

Язык имитационного моделирования GPSS создан специально для моделирования систем массового обслуживания и на данный момен является доминирующим в этой области. Однако, существующие версии систем имитационного моделирования на основе языка GPSS либо слишком дороги, либо ограничены в возможностях и не позволяют провести все необходимые исследования.\cite{KST} Помимо этого, на данный момент затруднено интегрирование моделей, разработанных при помощи GPSS в другие программные средства (напимер, в целях оптимизации параметров исследуемой системы).

Целью данной работы является создание системы имитационного моделирования, основанной на принципах и синтаксисе GPSS, однако позволяющей разрабатывать модели как часть более крупной программы.

В качестве языка разработки был выбран Haskell. Haskell является динамично развивающимся функциональным языком проограммирования, который получает все больше сторонников во всем мире, в том числе и в России. \cite{HaskellRef}. Для Haskell характерны строгая статическая типизация, модульность, строгое разделение функций на чистые и не чистые, ленивые вычисления, функции высших порядков и др.\cite{Haskell} Помимо этого использование языка Haskell позволит производить описание систем при помощи синтаксиса схожего с синтаксисом GPSS, при этом разработанные модели будут являться объектами первого класса, что позволит, например, передать модель как параметр в функцию оптимизации.

Для достижения поставленной цели необходимо решить следующие задачи:
\begin{itemize}
\item изучить принципы функционирования и синтаксис описания моделей в GPSS;
\item разработать синтаксис описания моделей схожий с синтаксисом GPSS, но при этом позволяющий составлять модели в виде функций языка Haskell;
\item выбрать подмножество блоков GPSS, которые следует реализовать в системе;
\item реализовать алгоритмы описания моделей и имитационного моделирования;
\item разработать и реализовать транслятор моделей GPSS в формат разработанной системы моделиования;
\item провести тестирование разработанного программного обеспечения;
\item провести моделирование некоторой эталонной системы массового обслуживания в разработанной системе, GPSS и аналитически и убедиться в совпадении полученных результатов.
\end{itemize}

