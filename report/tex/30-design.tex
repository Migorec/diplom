\chapter{Конструкторский раздел}

В данном разделе проводится выбор синтаксиса описания моделей в разрабатываемой системе, а также описываются алгоритмы и структуры данных, используемые при формировании моделей и непосредственно при моделировании. 



\section{}

Синтаксис разрабатываемой системы должен быть, на сколько это возможно, схож с синтаксисом системы GPSS. 

Программа на языке GPSS представляет из себя последовательность операторов, каждый из которы описывает тот или иной элемент модели (функцию, блок, устройство и др.). Этот подход естественен для императивных языков программирования, в которых программа является последовательностю комманд, меняющих состояние программы. Однако Haskell относится к категории функциональных языков, программы на которых лписываются как функции, значение которых вычисляется. При этом нет фиксированной, заданной программистом, последовательности операций, которые должны быть выполнены для достижения результата. 

Тем не менее, в языке Haskell предусмотрен механизм, позволяющий описать конкретную последовательность вычислений~--- монады. В сочетании с так называемой do-нотацией, этот механизм позволит проводить описание моделей на Haskel, используя синтаксис схожий с GPSS.

\section{Монады}

Понятие монады в языке Haskell основано на теории категорий. В рамках данной теории монадо может быть определена (не вполне строго) как моноид в категории эндофункторов. Однако для практического использования этого понятия в рамках языка Haskell можно обойтись менее формальным определением.

В соответствии с \cite{Haskell} монада~--- это контейнейрный тип данных (то есть такой, который содержит в себе значения других типов), представляющий собой экземпляр класса Monad определенного в модуле Prelude. 

Под классом в Haskell, понимается не тип данных, как в объектно-оринтированных языках, а набор методов (функций), которые применимы для работы с теми или иными типами данных, для которых объявлены экземпляры заданных классов. Наиболее близким аналогом классам в Haskell являются интерфейсы в таких языках как Java или C\#. Более точно их следует называть классами типов, но т.к. в данной работе используется исключительно функциональная парадигма, в дальнейшем для краткости они будут называться просто классами.

Класс Monad определен в модуле Prelude следующим образом:

\lstinputlisting[caption=Класс Monad,label=lst:monad]{inc/src/Monad.hs}

