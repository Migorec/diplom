\chapter{Технологический раздел}

В данном разделе описывается реализация разработанной системы, структура разработанных приложений и их тестирования.

\section{Выбор средств программной реализации}

Так как целью работы является разработка библиотеки для языка Haskell, целесообразно вести разработку на этом же языке. Язык Haskell относится к функциональным языкам общего назначения и обладает следующими особенностями:

\begin{itemize}
\item Чистые функции. Большинство функций в языке Haskell являются чистыми, то есть детерминированными и не имеющими побочных эффектов. При использовании таких функций программист может быть уверен, что при вызове такой функции не будет произведено каких-либо неявных действий (запись даных в файл, изменение значения переданных параметров, изменеие состояния некоторого объекта и т.п.) и на одних и тех же входны параметрах функкция всегда вернет одинаковый результат. Это существенно упрощает разраотку и тестирование программ.

\item Ленивые вычисления. Все вычисления в Haskell по умолчанию являются ленивыми, то есть ни одно значение не будет вычислено до тех пор, пока его значение действительно не понадобится. Этот механизм позволяет экономить выислительные мощности и работать со структурами вроде бесконечных списков. Однако при неаккураьном использовании это может привести к неэффективному расходованию памяти.

\item Строгая статическая типизация. Строгая типизация позволяет писать более надежные программы, так как несоответствие типов приведет к сообщению об ошибке, а не к некорректному поведению программы и будет быстрее обнаружено и исправлено. Статическая типизация позволяет выявить такие ошибки еще на этапе компиляции программы.

\item Автоматическое управление памятью. Как и большинство современных языков программирования Haskell берет на себя выделение и освобождение памяти. Это гарантировать отсутствие в разрабатываемой программе таких ошибок как переполнение буфера, неинициализированные переменные и т.п.

\end{itemize}

Помимо этого для Haskell существует обширный централизованный архив библиотек Hackage, и поисковая система Hoogle, позволяющая найти описание функции не ттолько по ее названи но и по сигнатуре.

\subsection{Построение графиков}

Для построения графиков была выбрана система Gnuplot. Это свободно распространяемая, кросплатформенная программа предназначенная для построения двух- и трехмерных графиков функций, заданных как аналитически, так и в виде наборов данных. Gnuplot поддержвает вывод результатов в различных форматах: растровых (PNG, JPEG), векторных (SVG, PDF), в виде кода LaTeX, в интерактивном режиме и др. Система используется для построения графиков в таких математических пакетах как GNU Octave, Maxima и других.

Для использования возможностей Gnuplot в программах на Haskell существует несколько библиотек, опубликованных на Hackage. В данной работе была использована библиотека EasyPlot.


\subsection{Построение пользовательского интерфейса}

Для построения пользовательского интерфейса демонстрационной программы была использована библиотека wxHaskell, которая в свою очередь является оберткой вокруг библиотеки для построения пользовательского интерфейса на C++ wxWidgets. 

Особенности wxWidgets:

\begin{itemize}
\item Созданные с помощью данной библиотеки приложения переносимы на большинство современных ОС.

\item В разработанном интерфейсе используются элементы управления привычные для пользователей целевой ОС. То есть стиль интерфейса программы будет отличаться на различных ОС и будет соответствовать рекомендуемому стилю для конкретной системы.

\end{itemize}

wxHaskell является надстройкой над wxWidgets, позволяющей создавать графический интерфейс к программам на языке Haskell. Она поддерживает большую часть функционала wxWidgets, позволяет описывать интерфейс в <<декларативном>> стиле с использованием функциональных связок и абстракцй высокого уровня. Библиотека особенно удобна для создания демонстрационных версий программ, так как во многом берет на себя решение адачи корректного расположения элементов управления на экране. 

\subsection{Сборка и развертывание библиотеки}

Для сборки разработанной библиотеки и развертывания ее на целевой машине была использована система Cabal. Данная система предоставляет единый интерфейс для создания и установки инсталяционных пакетов с программами и библиотеками на Haskell. Система связана с архивом библиотек Hackage и позволяет устанавливать хранящиеся там пакеты и оформить собственную программу в виде пакета для Hackage.

Информация о создаваемом пакете указывается в файле \Code{.cabal} в директории проекта. В нем указываются:
    
\begin{itemize}
\item имя пакета;

\item текущая версия;

\item информация об авторе и лицензии;

\item допустимые версии компилятора;

\item используемые расширения компилятора;

\item входящие в состав пакета библиотеки и исполняемые программы;

\item пакеты Hackage, необходимые для работы пакета;

\item и др.

\end{itemize}
    
    
Фрагмент файла \Code{.cabal} для разработаной библиотеки приведен ниже.

\lstinputlisting[caption=Фрагмент описания пакета,label=lst:cabal]{inc/src/.cabal}

Сборка и установка пакета выполняется следующими командами:

\begin{itemize}
\item \Code{cabal configure}~--- подготовка к сборке программа: определение целевой платформы, зависимостей и др.;

\item \Code{cabal build}~--- запуск процесса компиляции;

\item \Code{cabal install}~--- установка пакета в систему. Включает в себя первые две комманды;

\item \Code{cabal clean}~--- удаляет все временные файлы созданные предыдущими коммандами;

\end{itemize}
    
    
\section{Структура}

\begin{figure}[ht]
  \centering
  \includegraphics[width=\textwidth]{inc/dia/libStructHuge}
  \caption{Структура разработанной бибилиотеки}
  \label{fig:libStruct}
\end{figure}

